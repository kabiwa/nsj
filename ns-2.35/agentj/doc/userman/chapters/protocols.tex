\chapter{AgentJ Software Stack and Protocols}
\label{protocols}

This chapter describes the protocols that are implemented in Protolib

\section{Inter-\agentj~Communication}

Once an \agentj~node has been created and attached to its C++
counterpart, it can then use the supported \agentj~communication
and timing protocols in order to schedule events and
communicate with other \agentj~nodes. \agentj~nodes do this
by using \textbf{standard Java interfaces} which have been
re-implemented in order to bind to the simulation world within
NS2. These implementations include:

\begin{itemize}
\item \textbf{UDP Transport:} DatagramSocket and Multicast Socket have been
rebound to PAI and Protolib to work within NS2.
\item \textbf{TCP Transport:} under construction....
\item \textbf{Inet Support:} a number of the InetAddress methods have been
implemented to work within the NS2 context.
\item \textbf{Timing Libraries:} simple interfaces to timing functions have been
implemented so that non realtime events (i.e. NS timing events) can be triggered
at specific intervals during the simulation.
\end{itemize}

This functionality has been implemented by re-implementing the standard Java
interfaces to the networked Java counterparts of these functions. Therefore, in
order to use these within your Java program for example, you simply need to
import the \emph{pai.net} versions of DatagramSocket and MulticastSocket
rather than the \emph{java.io} versions.


Figure \ref{agentj:fig:javaFullOverview} gives an overview of the lower layers
of this integration and provides a complete picture of how \agentj~uses Java
within its implementation.  Here, the \agentj~nodes interface through the
Java PAI libraries by using a JNI binding to the C++ PAI and underlying
Protolib toolkits. It is this software stack that describes the full integration.
The standard Java interfaces described above merely provide convenient
access to these libraries.


Therefore, an \agentj~node interacts with other \agentj~nodes
by binding the standard Java interfaces to the PAI
socket and timing classes within the JNI implementation, as
indicated here. This integration allows \agentj~nodes to issue
commands to send data between NS2 nodes and to set up callbacks
for timing events within NS2.

For this implementation, a JNI interface is provided
between the Java PAI interface and the corresponding C++ PAI
interface that contains the necessary underlying
functionality. Within this implementation, the \agentj~node's
\emph{ID} is used along within the JNI binding to the
PAI interface in order to re-associate the Java object within its
C++ Ns2 agent when we return back to the C++ context.
In effect, what we are doing here is creating a JVM from C++,
then we are using JNI to re-enter C++.

However, when we do
this, JNI has no context and therefore we cannot use static
methods to obtain pointers to parts of the NS2 system. Therefore,
we have to transport these pointers manually through the JVM
so we can re-establish our context when we are back in the
C++ world. Specifically, the \agentj~node needs to be able to
reference its corresponding C++ NS2 agent so it can invoke
the calls in the appropriate way. Otherwise, how would Protolib
know which node to send the data from?