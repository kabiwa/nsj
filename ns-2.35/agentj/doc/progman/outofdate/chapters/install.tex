\chapter{Installing the \agentj~Toolkit}
\label{install}

This chapter describes the installation of core packages needed in
order to get the \agentj~toolkit operating. The core packages needed
are:

\index{Protolib Installation}

\begin{enumerate}
\item \textbf{Protolib:} a core package for adding timers and UDP
communication within NS2 \cite{protolib}
\item \textbf{\agentj:} this includes a customized version of the 
P2PS middleware \cite{p2ps} and the PAI interface to Protolib,
described in Chapt. \ref{pai} and the JNI interface for attaching  
Java Objects to NS2 nodes.
\end{enumerate}

\section{Downloading the Pieces} 

NS2 version 2.26 is the recommended version for use with \agentj.  It can be
downloaded as a file called \emph{ns-allinone-2.26.tar.gz} from
\emph{http://www.isi.edu/nsnam/dist/}.

The source code for \agentj~ and Protolib can be retrieved via CVS from the
Protean Forge research site hosted by the United States Naval Research
Laboratory at \emph{http://pf.itd.nrl.navy.mil/}.  Protolib is listed as its
own project on Protean Forge, but \agentj~ is listed within the SRSS project. 
These project pages contain bug lists, user forums, source code of major file
releases, and publically accessible CVS repositories of the latest development
source code. 

Protolib can be downloaded via CVS by running the following commands:

\begin{itemize}
\item{{\tt cvs -d :pserver:anonymous@protolib.pf.itd.nrl.navy.mil:/cvsroot/protolib login}}
\item{{\tt cvs -z3 -d :pserver:anonymous@protolib.pf.itd.nrl.navy.mil:/cvsroot/protolib co .}}
\end{itemize}

\agentj~ and P2PS-x can be retrieved via CVS by running the following commands:

\begin{itemize}
\item{{\tt cvs -d :pserver:anonymous@srss.pf.itd.nrl.navy.mil:/cvsroot/srss login}}
\item{{\tt cvs -z3 -d :pserver:anonymous@srss.pf.itd.nrl.navy.mil:/cvsroot/srss co agentj}}
\item{{\tt cvs -z3 -d :pserver:anonymous@srss.pf.itd.nrl.navy.mil:/cvsroot/srss co p2ps-x}}
\end{itemize}


The p2ps-x module contains classes required by \agentj.  After you've
successfully checked out these modules, the \agentj~ manual you are probably
currently reading will be in {\tt agentj/doc/agentj.pdf}.


\section{Installing the Protolib NS2 Binding}
\label{install:protolib}

With the Protolib release there is a supplied Makefile for NS version
2.26 and a README.TXT file in the \emph{ns} directory. The read me 
file describes the steps involved in installing protolob into NS2. They
are as follows:
 
\footnotesize
\begin{verbatim}
To use PROTOLIB with ns, you will need to at least
modify the ns "Makefile.in" to build the PROTOLIB
code into ns.  To do this, use the following steps:


1)  Make a link to the PROTOLIB source directory in the
    ns source directory.  (I use "protolib" for the link 
    name in the steps below). 

2)  Provide paths to the PROTOLIB include files by setting

    PROTOLIB_INCLUDES = -Iprotolib/common -Iprotolib/ns

    and adding $(PROTOLIB_INCLUDES) to the "INCLUDES" macro
    already defined in the  ns "Makefile.in" 

3)  Define compile-time CFLAGS needed for the PROTOLIB code
    by setting

    PROTOLIB_FLAGS = -DUNIX -DNS2 -DPROTO_DEBUG -DHAVE_ASSERT

    and adding $(PROTOLIB_FLAGS) to the "CFLAGS" macro already
    defined in the ns "Makefile.in"

4)  Add the list of PROTOLIB object files to get compiled and linked
    during the ns build.  For example, set

    OBJ_PROTOLIB_CPP = \
            protolib/ns/nsProtoAgent.o protolib/common/protoSim.o\
            protolib/common/networkAddress.o \ 
            protolib/common/protocolTimer.o \
            protolib/common/debug.o

    and then add $(OBJ_PROTOLIB_CPP) to the list in the "OBJ" 
    macro already defined in the ns "Makefile.in"

    Note: "nsProtoAgent.cpp" contains a starter ns agent which uses the
    PROTOLIB ProtocolTimer and UdpSocket classes.

5)  Add the the rule for .cpp files to ns-2 "Makefile.in":

    .cpp.o:
	    @rm -f $@
	    $(CC) -c $(CFLAGS) $(INCLUDES) -o $@ $*.cpp
        
    and add to the ns-2 Makefile.in "SRC" macro definition:
    
    $(OBJ_CPP:.o=.cpp)
    

6)  Run "./configure" in the ns source directory to create a new
    Makefile and  then type "make ns" to rebuild ns.
    
    
Brian Adamson
<mailto://adamson@itd.nrl.navy.mil>
18 December 2001
\end{verbatim}
\normalsize

The first thing to take note is that Protolib is basically a plug-in for NS
2 to allowing a trigger mechanism, based on timers and a UDP socket
implementation for passing data between NS2 nodes.  Therefore,
to install this plug-in, you must recompile NS2. It is therefore advisable
to build NS2 from scratch then add the protolib plug-in.

\index{\agentj~installation}

\section{Installing \agentj}
\label{install:agentj}

The \agentj~toolkit installation follows a similar installation path to 
Protolib but instead of using a softlink, it uses environment variables within 
the \emph{Makefile.in} file to point to the installation directory for the
source code for \agentj. The installation is provided below
and follows a similar style to the Protolib procedure for simplicity.
This file can be found in the src/build/ns2PAIConfig directory
within  the \agentj~source tree.

To install \agentj~ you need to first install 
Protolib and then modify the ns "Makefile.in" to build the \agentj
code into ns (there is a Makefile.in file for the ns2.26 
release given in the \agentj~build directory). Before, we
describe how to modify the NS2 makefile, let's take a brief look
at the environment variables I define that are used by \agentj and NS-2.

\section{Environment Variables}
\label{install:env}

In my \emph{.tcshrc} file, I define the following environment variables, which
allow me to specify the installation directories and specify class paths, library 
paths and NS-2 specifics in on file.  The environment variables here are 
defined in my \emph{.tcshrc} file on my MAC but could easily be converted 
to \emph{.bat} files (for windows) or csh or bash shells on Unix systems.

\footnotesize
\begin{verbatim}

setenv AGENTJ /Users/scmijt/Apps/nrl/agentj

setenv AGENTJDEBUG ON

setenv AGENTJXMLCONFIG $AGENTJ/config/AgentJConfig.xml

setenv PATH $NS/bin:/unix:$NS/tcl8.3.2/unix:$NS/tk8.3.2/unix:$PATH

setenv LD_LIBRARY_PATH $AGENTJ/lib/:$NS/otcl-1.0a8:$NS/lib

setenv TCL_LIBRARY $NS/tcl8.3.2/library
   
setenv CLASSPATH $AGENTJ/classes:$AGENTJ/lib/autolog.jar:
$AGENTJ/lib/log4j-1.2.8.jar

\end{verbatim}
\normalsize

For a Linux system, the CLASSPATH should also include paths to {\tt
p2ps-x/classes/}, {\tt p2ps-x/lib/}, and the jar files in {\tt p2ps-x/lib/}. 
As an example, I define the CLASSPATH in my bash environment on a Linux box as
follows:

\footnotesize
\begin{verbatim}
export CLASSPATH=/home/iandow/p2ps/development/p2ps-x/classes:\
/home/iandow/p2ps/development/agentj/classes:\
/home/iandow/p2ps/development/agentj//lib/autolog.jar:\
/home/iandow/p2ps/development/agentj//lib/log4j-1.2.8.jar:\
/home/iandow/p2ps/development/p2ps-x/lib/gap.jar:\
/home/iandow/p2ps/development/p2ps-x/lib/jdom.jar:\
/home/iandow/p2ps/development/p2ps-x/lib/p2ps.jar
\end{verbatim}
\normalsize

Descriptions of the purpose of of these definitions are given as follows:

\begin{itemize}
\item \textbf{AGENTJ:} is used to specify the installation directory of the agentj
package. This is used by the Makefile.in NS-2 makefile and also used within the other environment variables defined here.

\item \textbf{AGENTJDEBUG:} is used to specify whether you want to turn on logging or not.  Within the C++ parts of the code, we use a simple custom logging scheme, whereas within the Java parts of the code, log4j is used.  To turn on logging throughout the system, set this environment variable to 'ON'. Any other value (or no definition of this variable) will resort to the default setting i.e. no debugging information will be displayed.  Logging is described in more detail in Section \ref{install:logging}.

\item \textbf{AGENTJXMLCONFIG:} This environment variable allows you to specify the format for the log4j java logging using a log4j XML file.  See the log4j web site \cite{log4j} for more information on how to specify these.  An example configuration is supplied in the \emph{config} directory, called AgentJConfig.xml, as indicated.

\item \textbf{PATH:} the standard PATH variable for specifying directories that contain executables.  Here, I simply include the directories required by NS-2 version 2.26.  For more information, see \cite{ns2} 

\item \textbf{LD\_LIBRARY\_PATH:} the standard environment variable for specifying where to find libraries.  Here, I extend this to include the \agentj  lib directory plus some directories required by NS2, version 2.26.

\item \textbf{TCL\_LIBRARY:} required for NS2 installation

\item \textbf{CLASSPATH:} the standard environment variable used to specify the Java classpath.  Here, I extend this with the JAR files and directories required by Agentj e.g.  
  agentj classes directory and two JAR files required for the Java logging: autolog.jar and  log4j-1.2.8.jar.
\end{itemize}

\section{Installation into NS2}
\label{install:install}

To install \agentj, use the following steps:

1)  Install Protolib

2)  Set the AGENTJ environment variable to point to your installation 
directory for \agentj~ and create pointers to the various subdirectories 
for the source in the Makfile.in NS2 file, as follows:

\footnotesize
\begin{verbatim}

AGENTJ_SRC = $(AGENTJ)/src/c
AGENTJ_LIB_DIR = $(AGENTJ)/lib

AGENTJ_C_SRC = $(AGENTJ_SRC)/agentj
AGENTJ_UTILS = $(AGENTJ_SRC)/utils
PAI = $(AGENTJ_SRC)/pai
PAI_IMP = $(PAI)/imp
PAI_API = $(PAI)/api
PAI_AGENT = $(PAI_IMP)/agent
PAI_FACTORY = $(PAI_IMP)/factory
PAI_FACTORY_NET = $(PAI_FACTORY)/net
PAI_FACTORY_NS = $(PAI_FACTORY)/ns
PAI_IMP_JNI = $(PAI_IMP)/jni

2)  Provide paths to the AGENTJ include files by setting

AGENTJ_INCLUDES = -I$(JAVA_HOME)/include -I$(AGENTJ_C_SRC) -I$(AGENTJ_UTILS) -I$(PAI) -I$(PAI_AGENT) -I$(PAI_API) -I$(PAI_FACTORY) -I$(PAI_FACTORY_NET) -I$(PAI_FACTORY_NS) -I$(PAI_IMP_JNI) # -I$(LOG4CPLUS_INCLUDE_DIR)

    and adding $(AGENTJ_INCLUDES) to the "INCLUDES" macro
    already defined in the  ns "Makefile.in"
    
    On Linux platforms, include the path to jni_md.h in the list of AGENTJ 
    include files.  So, AGENTJ_INCLUDES might look like something like this:

AGENTJ_INCLUDES = -I$(JAVA_HOME)/include -I$(JAVA_HOME)/include/linux -I$(AGENTJ_C_SRC) -I$(AGENTJ_UTILS) -I$(PAI) -I$(PAI_AGENT) -I$(PAI_API) -I$(PAI_FACTORY) -I$(PAI_FACTORY_NET) -I$(PAI_FACTORY_NS) -I$(PAI_IMP_JNI) 
    

3)  Add the list of AGENTJ object files to get compiled and linked
    during the ns build.  For example, set

OBJ_AGENTJ_CPP = $(AGENTJ_UTILS)/LinkedList.o $(PAI_FACTORY)/PAIDispatcher.o \
	$(PAI_FACTORY)/PAIEngine.o $(PAI_FACTORY)/PAIFactory.o \
	$(PAI_FACTORY)/PAIMultipleListener.o $(PAI_FACTORY)/PAISocket.o \
	$(PAI_FACTORY)/PAITimer.o $(PAI_FACTORY)/PAIEnvironment.o \
	$(PAI_FACTORY)/PAIListener.o \
	$(PAI_FACTORY_NS)/PAINS2UDPSocket.o \
	$(PAI_FACTORY_NS)/PAINS2Timer.o \
	$(PAI_API)/PAI.o \
	$(PAI_AGENT)/PAIAgent.o $(PAI_AGENT)/PAISimpleAgent.o \
	$(AGENTJ_C_SRC)/C2JBroker.o $(AGENTJ_C_SRC)/Agentj.o \
	$(PAI_IMP_JNI)/JNIBridge.o $(PAI_IMP_JNI)/JNIImp.o

    and then add $(OBJ_AGENTJ_CPP) to the list in the "OBJ" 
macro already defined in the ns "Makefile.in"

    Note: "Agentj.cpp" contains the NS agent for integrating Java objects.

4)  Add the rule for .cpp files to ns-2 "Makefile.in":

    .cpp.o:
	    @rm -f $@
	    $(CC) -c $(CFLAGS) $(INCLUDES) -o $@ $*.cpp
        
    and add to the ns-2 Makefile.in "SRC" macro definition:
    
    $(OBJ_CPP:.o=.cpp)

    (note this has already been done - if you have installed 
protolib correctly).

5)  Create a shared library - define compile-time SHARED 
Library flags and  libraries needed for your platform to 
create a shared library (this is needed for the JNI binding). 
On my Mac OS 10.x, these are defined as follows:

AGENTJ_LIB = -framework JavaVM 
AGENTJ_SHARED_LDFLAGS = -dynamiclib -lresolv

    and adding $(AGENTJ_LIB) to the "LIB" macro already
    defined in the ns "Makefile.in"

    and adding a new rule to make the shared library and 
    put it in the correct place:

libagentj.jnilib: $(OBJ) common/tclAppInit.o
	$(LINK) $(AGENTJ_SHARED_LDFLAGS) -o $@ \
		common/tclAppInit.o $(OBJ) $(LIB)
	mv libagentj.jnilib $(AGENTJ_LIB_DIR)
    

On a Linux box running a 2.6.7 kernel with version 1.5.0 of Sun's JDK , these
flags are defined as follows:

AGENTJ_LIB = -L$(JAVA_HOME)/jre/lib/i386/server/ -ljvm
AGENTJ_SHARED_LDFLAGS = -shared

    Still add $(AGENTJ_LIB) to the "LIB" macro already
    defined in the ns "Makefile.in", just as for Macs.

    The new rule for making the shared library for Linux 
    should look like this:

libagentj.so: $(OBJ) common/tclAppInit.o
	$(LINK) $(AGENTJ_SHARED_LDFLAGS) -o $@ common/tclAppInit.o $(OBJ) $(LIB)
	mv libagentj.so $(AGENTJ_LIB_DIR)



6)  Run "./configure" in the ns source directory to create a new
    Makefile 

7) Type "make ns" to rebuild ns - this creates the static library
      
8) Type "make libagentj.jnilib" (or "make libagentj.so" for Linux platforms) to
make the dynamic library  needed for the installation of the JNI frameworks.
\end{verbatim} \normalsize



\subsection{The NS-2 Makefile for Macintosh}
\label{install:p2ps-ns2-build}

The resulting NS2 Makefile should therefore including both the
Protolib and \agentj~dependencies. A complete version of my 
Makefile, used to build NS 2 version 2.26 on an Apple Mac, 
is provided below:

\footnotesize
\begin{verbatim}
#  Copyright (c) 1994, 1995, 1996
# 	The Regents of the University of California.  All rights reserved.
#
#  Redistribution and use in source and binary forms, with or without
#  modification, are permitted provided that: (1) source code distributions
#  retain the above copyright notice and this paragraph in its entirety, (2)
#  distributions including binary code include the above copyright notice and
#  this paragraph in its entirety in the documentation or other materials
#  provided with the distribution, and (3) all advertising materials mentioning
#  features or use of this software display the following acknowledgement:
#  ``This product includes software developed by the University of California,
#  Lawrence Berkeley Laboratory and its contributors.'' Neither the name of
#  the University nor the names of its contributors may be used to endorse
#  or promote products derived from this software without specific prior
#  written permission.
#  THIS SOFTWARE IS PROVIDED ``AS IS'' AND WITHOUT ANY EXPRESS OR IMPLIED
#  WARRANTIES, INCLUDING, WITHOUT LIMITATION, THE IMPLIED WARRANTIES OF
#  MERCHANTABILITY AND FITNESS FOR A PARTICULAR PURPOSE.
#
# @(#) $Header: 2002/10/09 15:34:11

#
# Various configurable paths (remember to edit Makefile.in, not Makefile)
#

# Top level hierarchy
prefix	= @prefix@
# Pathname of directory to install the binary
BINDEST	= @prefix@/bin
# Pathname of directory to install the man page
MANDEST	= @prefix@/man

BLANK	= # make a blank space.  DO NOT add anything to this line

# The following will be redefined under Windows (see WIN32 lable below)
CC	= @CC@
CPP	= @CXX@
LINK	= $(CPP)
MKDEP	= ./conf/mkdep
TCLSH	= @V_TCLSH@
TCL2C	= @V_TCL2CPP@
AR	= ar rc $(BLANK)

RANLIB	= @V_RANLIB@
INSTALL	= @INSTALL@
LN	= ln
TEST	= test
RM	= rm -f
MV      = mv
PERL	= @PERL@

# for diffusion
#DIFF_INCLUDES = "./diffusion3/main ./diffusion3/lib ./diffusion3/nr ./diffusion3/ns"

# Flags for creating a shared library - IANS Additions
 
CCOPT	= @V_CCOPT@
STATIC	= @V_STATIC@
LDFLAGS	= $(STATIC)
LDOUT	= -o $(BLANK)

########################### New Protolib Section #######################

OBJ_PROTOLIB_CPP = \
        protolib/ns/nsProtoSimAgent.o protolib/common/protoSimAgent.o \
        protolib/ns/nsProtoRouteMgr.o \
        protolib/common/protoSimSocket.o protolib/common/protoAddress.o \
        protolib/common/protoTimer.o protolib/common/protoExample.o \
        protolib/common/protoDebug.o protolib/common/protoRouteMgr.o \
        protolib/common/protoRouteTable.o protolib/common/protoTree.o

########################### Protolib Section #######################

PROTOLIB = ../../protolib

PROTOLIB_INCLUDES = -I$(PROTOLIB)/common -I$(PROTOLIB)/ns

PROTOLIB_FLAGS = -DNS2 -DSIMULATE -DUNIX -DPROTO_DEBUG -DHAVE_ASSERT -DHAVE_DIRFD

DEFINE	= -DTCP_DELAY_BIND_ALL -DNO_TK @V_DEFINE@ 
@V_DEFINES@ @DEFS@ -DNS_DIFFUSION 
-DSMAC_NO_SYNC -DSTL_NAMESPACE=@STL_NAMESPACE@ 
-DUSE_SINGLE_ADDRESS_SPACE

OBJ_PROTOLIB_CPP = \
            $(PROTOLIB)/ns/nsProtoAgent.o $(PROTOLIB)/common/protoSim.o \
            $(PROTOLIB)/common/networkAddress.o $(PROTOLIB)/common/protocolTimer.o \
            $(PROTOLIB)/common/debug.o

############################## AGENTJ Section ########################
 
# YOU MUST SPECIFY THE AGENTJ environment variable or set this directly here

AGENTJ_SRC = $(AGENTJ)/src/c
AGENTJ_LIB_DIR = $(AGENTJ)/lib

AGENTJ_C_SRC = $(AGENTJ_SRC)/agentj
AGENTJ_UTILS = $(AGENTJ_SRC)/utils
PAI = $(AGENTJ_SRC)/pai
PAI_IMP = $(PAI)/imp
PAI_API = $(PAI)/api
PAI_AGENT = $(PAI_IMP)/agent
PAI_FACTORY = $(PAI_IMP)/factory
PAI_FACTORY_NET = $(PAI_FACTORY)/net
PAI_FACTORY_NS = $(PAI_FACTORY)/ns
PAI_IMP_JNI = $(PAI_IMP)/jni

#Note just include the ns implementation here - NOT the net directory

AGENTJ_LIB = -framework JavaVM 
AGENTJ_SHARED_LDFLAGS = -dynamiclib -lresolv

AGENTJ_INCLUDES = -I$(JAVA_HOME)/include -I$(AGENTJ_C_SRC) -I$(AGENTJ_UTILS) -I$(PAI) -I$(PAI_AGENT) -I$(PAI_API) -I$(PAI_FACTORY) -I$(PAI_FACTORY_NET) -I$(PAI_FACTORY_NS) -I$(PAI_IMP_JNI) # -I$(LOG4CPLUS_INCLUDE_DIR)

OBJ_AGENTJ_CPP = $(AGENTJ_UTILS)/LinkedList.o $(PAI_FACTORY)/PAIDispatcher.o \
	$(PAI_FACTORY)/PAIEngine.o $(PAI_FACTORY)/PAIFactory.o \
	$(PAI_FACTORY)/PAIMultipleListener.o $(PAI_FACTORY)/PAISocket.o \
	$(PAI_FACTORY)/PAITimer.o $(PAI_FACTORY)/PAIEnvironment.o \
	$(PAI_FACTORY)/PAIListener.o \
	$(PAI_FACTORY_NS)/PAINS2UDPSocket.o \
	$(PAI_FACTORY_NS)/PAINS2Timer.o \
	$(PAI_API)/PAI.o \
	$(PAI_AGENT)/PAIAgent.o $(PAI_AGENT)/PAISimpleAgent.o \
	$(AGENTJ_C_SRC)/C2JBroker.o $(AGENTJ_C_SRC)/Agentj.o \
	$(PAI_IMP_JNI)/JNIBridge.o $(PAI_IMP_JNI)/JNIImp.o


######################## END AGENTJ Section ############################

INCLUDES = \
	$(PROTOLIB_INCLUDES) \
	$(AGENTJ_INCLUDES) \
	-I. @V_INCLUDE_X11@ \
	@V_INCLUDES@ \
	-I./tcp -I./common -I./link -I./queue \
	-I./adc -I./apps -I./mac -I./mobile -I./trace \
	-I./routing -I./tools -I./classifier -I./mcast \
	-I./diffusion3/lib/main -I./diffusion3/lib \
	-I./diffusion3/lib/nr -I./diffusion3/ns \
	-I./diffusion3/diffusion -I./asim/ -I./qs


LIB	= \
	@V_LIBS@ \
	@V_LIB_X11@ \
	@V_LIB@ \
	$(AGENTJ_LIB) \
	-lm @LIBS@
#	-L@libdir@ \

CFLAGS	= $(CCOPT) $(DEFINE) $(PROTOLIB_FLAGS) $(AGENTJ_FLAGS)

# Explicitly define compilation rules since SunOS 4's make doesn't like gcc.
# Also, gcc does not remove the .o before forking 'as', which can be a
# problem if you don't own the file but can write to the directory.
.SUFFIXES: .cc	# $(.SUFFIXES)

.cc.o:
	@rm -f $@
	$(CPP) -c $(CFLAGS) $(INCLUDES) -o $@ $*.cc

.c.o:
	@rm -f $@
	$(CC) -c $(CFLAGS) $(INCLUDES) -o $@ $*.c


GEN_DIR	= gen/
LIB_DIR	= lib/
NS	= ns
NSX	= nsx
NSE	= nse

# To allow conf/makefile.win overwrite this macro
# We will set these two macros to empty in conf/makefile.win since VC6.0
# does not seem to support the STL in gcc 2.8 and up. 
OBJ_STL = diffusion3/lib/nr/nr.o diffusion3/lib/dr.o \
	diffusion3/ns/diffagent.o diffusion3/ns/diffrtg.o \
	diffusion3/ns/difftimer.o \
	diffusion3/diffusion/diffusion.o \
	diffusion3/lib/main/attrs.o \
	diffusion3/lib/main/iodev.o \
	diffusion3/lib/main/timers.o \
	diffusion3/lib/main/events.o \
	diffusion3/lib/main/message.o \
	diffusion3/lib/main/stats.o \
	diffusion3/lib/main/tools.o \
	diffusion3/lib/drivers/rpc_stats.o \
	diffusion3/apps/sysfilters/gradient.o \
	diffusion3/apps/sysfilters/log.o \
	diffusion3/apps/sysfilters/tag.o \
	diffusion3/apps/sysfilters/srcrt.o \
	diffusion3/lib/diffapp.o \
	diffusion3/apps/pingapp/ping_sender.o \
	diffusion3/apps/pingapp/ping_receiver.o \
	diffusion3/apps/pingapp/ping_common.o \
	diffusion3/apps/pingapp/push_receiver.o \
	diffusion3/apps/pingapp/push_sender.o \
	diffusion3/apps/gear/geo-attr.o \
	diffusion3/apps/gear/geo-routing.o \
	diffusion3/apps/gear/geo-tools.o \
	nix/hdr_nv.o nix/classifier-nix.o \
	nix/nixnode.o nix/nixvec.o \
	nix/nixroute.o

NS_TCL_LIB_STL = tcl/lib/ns-diffusion.tcl 



# WIN32: uncomment the following line to include specific make for VC++
# !include <conf/makefile.win>

OBJ_CC = \
	tools/random.o tools/rng.o tools/ranvar.o common/misc.o common/timer-handler.o \
	common/scheduler.o common/object.o common/packet.o \
	common/ip.o routing/route.o common/connector.o common/ttl.o \
	trace/trace.o trace/trace-ip.o \
	classifier/classifier.o classifier/classifier-addr.o \
	classifier/classifier-hash.o \
	classifier/classifier-virtual.o \
	classifier/classifier-mcast.o \
	classifier/classifier-bst.o \
	classifier/classifier-mpath.o mcast/replicator.o \
	classifier/classifier-mac.o \
	classifier/classifier-qs.o \
	classifier/classifier-port.o src_rtg/classifier-sr.o \
        src_rtg/sragent.o src_rtg/hdr_src.o adc/ump.o \
	qs/qsagent.o qs/hdr_qs.o \
	apps/app.o apps/telnet.o tcp/tcplib-telnet.o \
	tools/trafgen.o trace/traffictrace.o tools/pareto.o \
	tools/expoo.o tools/cbr_traffic.o \
	adc/tbf.o adc/resv.o adc/sa.o tcp/saack.o \
	tools/measuremod.o adc/estimator.o adc/adc.o adc/ms-adc.o \
	adc/timewindow-est.o adc/acto-adc.o \
        adc/pointsample-est.o adc/salink.o adc/actp-adc.o \
	adc/hb-adc.o adc/expavg-est.o\
	adc/param-adc.o adc/null-estimator.o \
	adc/adaptive-receiver.o apps/vatrcvr.o adc/consrcvr.o \
	common/agent.o common/message.o apps/udp.o \
	common/session-rtp.o apps/rtp.o tcp/rtcp.o \
	common/ivs.o \
	tcp/tcp.o tcp/tcp-sink.o tcp/tcp-reno.o \
	tcp/tcp-newreno.o \
	tcp/tcp-vegas.o tcp/tcp-rbp.o tcp/tcp-full.o tcp/rq.o \
	baytcp/tcp-full-bay.o baytcp/ftpc.o baytcp/ftps.o \
	tcp/scoreboard.o tcp/scoreboard-rq.o tcp/tcp-sack1.o tcp/tcp-fack.o \
	tcp/tcp-asym.o tcp/tcp-asym-sink.o tcp/tcp-fs.o \
	tcp/tcp-asym-fs.o tcp/tcp-qs.o \
	tcp/tcp-int.o tcp/chost.o tcp/tcp-session.o \
	tcp/nilist.o \
	tools/integrator.o tools/queue-monitor.o \
	tools/flowmon.o tools/loss-monitor.o \
	queue/queue.o queue/drop-tail.o \
	adc/simple-intserv-sched.o queue/red.o \
	queue/semantic-packetqueue.o queue/semantic-red.o \
	tcp/ack-recons.o \
	queue/sfq.o queue/fq.o queue/drr.o queue/srr.o queue/cbq.o \
	queue/jobs.o queue/marker.o queue/demarker.o \
	link/hackloss.o queue/errmodel.o queue/fec.o\
	link/delay.o tcp/snoop.o \
	gaf/gaf.o \
	link/dynalink.o routing/rtProtoDV.o common/net-interface.o \
	mcast/ctrMcast.o mcast/mcast_ctrl.o mcast/srm.o \
	common/sessionhelper.o queue/delaymodel.o \
	mcast/srm-ssm.o mcast/srm-topo.o \
	apps/mftp.o apps/mftp_snd.o apps/mftp_rcv.o \
	apps/codeword.o \
	routing/alloc-address.o routing/address.o \
	$(LIB_DIR)int.Vec.o $(LIB_DIR)int.RVec.o \
	$(LIB_DIR)dmalloc_support.o \
	webcache/http.o webcache/tcp-simple.o webcache/pagepool.o \
	webcache/inval-agent.o webcache/tcpapp.o webcache/http-aux.o \
	webcache/mcache.o webcache/webtraf.o \
	webcache/webserver.o \
	webcache/logweb.o \
	empweb/empweb.o \
	empweb/empftp.o \
	realaudio/realaudio.o \
	mac/lanRouter.o classifier/filter.o \
	common/pkt-counter.o \
	common/Decapsulator.o common/Encapsulator.o \
	common/encap.o \
	mac/channel.o mac/mac.o mac/ll.o mac/mac-802_11.o \
	mac/mac-802_3.o mac/mac-tdma.o mac/smac.o \
	mobile/mip.o mobile/mip-reg.o mobile/gridkeeper.o \
	mobile/propagation.o mobile/tworayground.o \
	mobile/antenna.o mobile/omni-antenna.o \
	mobile/shadowing.o mobile/shadowing-vis.o mobile/dumb-agent.o \
	common/bi-connector.o common/node.o \
	common/mobilenode.o \
	mac/arp.o mobile/god.o mobile/dem.o \
	mobile/topography.o mobile/modulation.o \
	queue/priqueue.o queue/dsr-priqueue.o \
	mac/phy.o mac/wired-phy.o mac/wireless-phy.o \
	mac/mac-timers.o trace/cmu-trace.o mac/varp.o \
	dsdv/dsdv.o dsdv/rtable.o queue/rtqueue.o \
	routing/rttable.o \
	imep/imep.o imep/dest_queue.o imep/imep_api.o \
	imep/imep_rt.o imep/rxmit_queue.o imep/imep_timers.o \
	imep/imep_util.o imep/imep_io.o \
	tora/tora.o tora/tora_api.o tora/tora_dest.o \
	tora/tora_io.o tora/tora_logs.o tora/tora_neighbor.o \
	dsr/dsragent.o dsr/hdr_sr.o dsr/mobicache.o dsr/path.o \
	dsr/requesttable.o dsr/routecache.o dsr/add_sr.o \
	dsr/dsr_proto.o dsr/flowstruct.o dsr/linkcache.o \
	dsr/simplecache.o dsr/sr_forwarder.o \
	aodv/aodv_logs.o aodv/aodv.o \
	aodv/aodv_rtable.o aodv/aodv_rqueue.o \
	common/ns-process.o \
	satellite/satgeometry.o satellite/sathandoff.o \
	satellite/satlink.o satellite/satnode.o \
	satellite/satposition.o satellite/satroute.o \
	satellite/sattrace.o \
	rap/raplist.o rap/rap.o rap/media-app.o rap/utilities.o \
	common/fsm.o tcp/tcp-abs.o \
	diffusion/diffusion.o diffusion/diff_rate.o diffusion/diff_prob.o \
	diffusion/diff_sink.o diffusion/flooding.o diffusion/omni_mcast.o \
	diffusion/hash_table.o diffusion/routing_table.o diffusion/iflist.o \
	tcp/tfrc.o tcp/tfrc-sink.o mobile/energy-model.o apps/ping.o tcp/tcp-rfc793edu.o \
	queue/rio.o queue/semantic-rio.o tcp/tcp-sack-rh.o tcp/scoreboard-rh.o \
	plm/loss-monitor-plm.o plm/cbr-traffic-PP.o \
	linkstate/hdr-ls.o \
	mpls/classifier-addr-mpls.o mpls/ldp.o mpls/mpls-module.o \
	routing/rtmodule.o classifier/classifier-hier.o \
	routing/addr-params.o \
	routealgo/rnode.o \
	routealgo/bfs.o \
	routealgo/rbitmap.o \
	routealgo/rlookup.o \
	routealgo/routealgo.o \
	diffserv/dsred.o diffserv/dsredq.o \
	diffserv/dsEdge.o diffserv/dsCore.o \
	diffserv/dsPolicy.o diffserv/ew.o\
	queue/red-pd.o queue/pi.o queue/vq.o queue/rem.o \
	queue/gk.o \
	pushback/rate-limit.o pushback/rate-limit-strategy.o \
	pushback/ident-tree.o pushback/agg-spec.o \
	pushback/logging-data-struct.o \
	pushback/rate-estimator.o \
	pushback/pushback-queue.o pushback/pushback.o \
	common/parentnode.o trace/basetrace.o \
	common/simulator.o asim/asim.o \
	common/scheduler-map.o common/splay-scheduler.o \
	linkstate/ls.o linkstate/rtProtoLS.o \
	pgm/classifier-pgm.o pgm/pgm-agent.o pgm/pgm-sender.o \
	pgm/pgm-receiver.o mcast/rcvbuf.o \
	mcast/classifier-lms.o mcast/lms-agent.o mcast/lms-receiver.o \
	mcast/lms-sender.o \
	@V_STLOBJ@


# don't allow comments to follow continuation lines

#  mac-csma.o mac-multihop.o\
#	sensor-nets/landmark.o mac-simple-wireless.o \
#	sensor-nets/tags.o sensor-nets/sensor-query.o \
#	sensor-nets/flood-agent.o \

# what was here before is now in emulate/
OBJ_C =

OBJ_COMPAT = $(OBJ_GETOPT) common/win32.o
#XXX compat/win32x.o compat/tkConsole.o

OBJ_EMULATE_CC = \
	emulate/net-ip.o \
	emulate/net.o \
	emulate/tap.o \
	emulate/ether.o \
	emulate/internet.o \
	emulate/ping_responder.o \
	emulate/arp.o \
	emulate/icmp.o \
	emulate/net-pcap.o \
	emulate/nat.o  \
	emulate/iptap.o \
	emulate/tcptap.o

OBJ_EMULATE_C = \
	emulate/inet.o

OBJ_GEN = $(GEN_DIR)version.o $(GEN_DIR)ns_tcl.o $(GEN_DIR)ptypes.o

SRC =	$(OBJ_C:.o=.c) $(OBJ_CC:.o=.cc) \
	$(OBJ_EMULATE_C:.o=.c) $(OBJ_EMULATE_CC:.o=.cc) \
	$(OBJ_CPP:.o=.cpp) \
	common/tclAppInit.cc common/tkAppInit.cc 

OBJ =	$(OBJ_C) $(OBJ_CC) $(OBJ_GEN) $(OBJ_COMPAT) $(OBJ_PROTOLIB_CPP) $(OBJ_AGENTJ_CPP)

CLEANFILES = ns nse nsx ns.dyn $(OBJ) $(OBJ_EMULATE_CC) \
	$(OBJ_EMULATE_C) common/tclAppInit.o \
	$(GEN_DIR)* $(NS).core core core.$(NS) core.$(NSX) core.$(NSE) \
	common/ptypes2tcl common/ptypes2tcl.o 

SUBDIRS=\
	indep-utils/cmu-scen-gen/setdest \
	indep-utils/webtrace-conv/dec \
	indep-utils/webtrace-conv/epa \
	indep-utils/webtrace-conv/nlanr \
	indep-utils/webtrace-conv/ucb

BUILD_NSE = @build_nse@

all: $(NS) $(BUILD_NSE) all-recursive


all-recursive:
	for i in $(SUBDIRS); do ( cd $$i; $(MAKE) all; ) done

$(NS): $(OBJ) common/tclAppInit.o Makefile
	$(LINK) $(LDFLAGS) $(LDOUT)$@ \
		common/tclAppInit.o $(OBJ) $(LIB)

Makefile: Makefile.in
	@echo "Makefile.in is newer than Makefile."
	@echo "You need to re-run configure."
	false

$(NSE): $(OBJ) common/tclAppInit.o $(OBJ_EMULATE_CC) $(OBJ_EMULATE_C)
	$(LINK) $(LDFLAGS) $(LDOUT)$@ \
		common/tclAppInit.o $(OBJ) \
		$(OBJ_EMULATE_CC) $(OBJ_EMULATE_C)  $(LIB) 

ns.dyn: $(OBJ) common/tclAppInit.o
	$(LINK) $(LDFLAGS) -o $@ \
		common/tclAppInit.o $(OBJ) $(LIB)

libagentj.jnilib: $(OBJ) common/tclAppInit.o
	$(LINK) $(AGENTJ_SHARED_LDFLAGS) -o $@ \
		common/tclAppInit.o $(OBJ) $(LIB)
	mv libagentj.jnilib $(AGENTJ_LIB_DIR)

PURIFY	= purify -cache-dir=/tmp
ns-pure: $(OBJ) common/tclAppInit.o
	$(PURIFY) $(LINK) $(LDFLAGS) -o $@ \
		common/tclAppInit.o $(OBJ) $(LIB)

NS_TCL_LIB = \
	tcl/lib/ns-compat.tcl \
	tcl/lib/ns-default.tcl \
	tcl/lib/ns-errmodel.tcl \
	tcl/lib/ns-lib.tcl \
	tcl/lib/ns-link.tcl \
	tcl/lib/ns-mobilenode.tcl \
	tcl/lib/ns-sat.tcl \
	tcl/lib/ns-cmutrace.tcl \
	tcl/lib/ns-node.tcl \
	tcl/lib/ns-rtmodule.tcl \
	tcl/lib/ns-hiernode.tcl \
	tcl/lib/ns-packet.tcl \
	tcl/lib/ns-queue.tcl \
	tcl/lib/ns-source.tcl \
	tcl/lib/ns-nam.tcl \
	tcl/lib/ns-trace.tcl \
	tcl/lib/ns-agent.tcl \
	tcl/lib/ns-random.tcl \
	tcl/lib/ns-namsupp.tcl \
	tcl/lib/ns-address.tcl \
	tcl/lib/ns-intserv.tcl \
	tcl/lib/ns-autoconf.tcl \
	tcl/rtp/session-rtp.tcl \
	tcl/lib/ns-mip.tcl \
	tcl/rtglib/dynamics.tcl \
	tcl/rtglib/route-proto.tcl \
	tcl/rtglib/algo-route-proto.tcl \
	tcl/rtglib/ns-rtProtoLS.tcl \
        tcl/interface/ns-iface.tcl \
	tcl/mcast/BST.tcl \
        tcl/mcast/ns-mcast.tcl \
        tcl/mcast/McastProto.tcl \
        tcl/mcast/DM.tcl \
	tcl/mcast/srm.tcl \
	tcl/mcast/srm-adaptive.tcl \
	tcl/mcast/srm-ssm.tcl \
	tcl/mcast/timer.tcl \
	tcl/mcast/McastMonitor.tcl \
	tcl/mcast/mftp_snd.tcl \
	tcl/mcast/mftp_rcv.tcl \
	tcl/mcast/mftp_rcv_stat.tcl \
	tcl/mobility/dsdv.tcl \
	tcl/mobility/dsr.tcl \
        tcl/ctr-mcast/CtrMcast.tcl \
        tcl/ctr-mcast/CtrMcastComp.tcl \
        tcl/ctr-mcast/CtrRPComp.tcl \
	tcl/rlm/rlm.tcl \
	tcl/rlm/rlm-ns.tcl \
	tcl/session/session.tcl \
	tcl/lib/ns-route.tcl \
	tcl/emulate/ns-emulate.tcl \
	tcl/lan/vlan.tcl \
	tcl/lan/abslan.tcl \
	tcl/lan/ns-ll.tcl \
	tcl/lan/ns-mac.tcl \
	tcl/webcache/http-agent.tcl \
	tcl/webcache/http-server.tcl \
	tcl/webcache/http-cache.tcl \
	tcl/webcache/http-mcache.tcl \
	tcl/webcache/webtraf.tcl \
	tcl/webcache/empweb.tcl \
	tcl/webcache/empftp.tcl \
	tcl/plm/plm.tcl \
	tcl/plm/plm-ns.tcl \
	tcl/plm/plm-topo.tcl \
	tcl/mpls/ns-mpls-classifier.tcl \
	tcl/mpls/ns-mpls-ldpagent.tcl \
	tcl/mpls/ns-mpls-node.tcl \
	tcl/mpls/ns-mpls-simulator.tcl \
	tcl/lib/ns-pushback.tcl \
	tcl/lib/ns-srcrt.tcl \
	tcl/mcast/ns-lms.tcl \
	tcl/lib/ns-qsnode.tcl \
	@V_NS_TCL_LIB_STL@

$(GEN_DIR)ns_tcl.cc: $(NS_TCL_LIB)
	$(TCLSH) bin/tcl-expand.tcl tcl/lib/ns-lib.tcl @V_NS_TCL_LIB_STL@ | $(TCL2C) et_ns_lib > $@

$(GEN_DIR)version.c: VERSION
	$(RM) $@
	$(TCLSH) bin/string2c.tcl version_string < VERSION > $@

$(GEN_DIR)ptypes.cc: common/ptypes2tcl common/packet.h
	./common/ptypes2tcl > $@

common/ptypes2tcl: common/ptypes2tcl.o
	$(LINK) $(LDFLAGS) $(LDOUT)$@ common/ptypes2tcl.o

common/ptypes2tcl.o: common/ptypes2tcl.cc common/packet.h

install: force install-ns install-man install-recursive

install-ns: force
	$(INSTALL) -m 555 -o bin -g bin ns $(DESTDIR)$(BINDEST)

install-man: force
	$(INSTALL) -m 444 -o bin -g bin ns.1 $(DESTDIR)$(MANDEST)/man1

install-recursive: force
	for i in $(SUBDIRS); do ( cd $$i; $(MAKE) install; ) done

clean:
	$(RM) $(CLEANFILES)

AUTOCONF_GEN = tcl/lib/ns-autoconf.tcl
distclean: distclean-recursive
	$(RM) $(CLEANFILES) Makefile config.cache config.log config.status \
	    autoconf.h gnuc.h os-proto.h $(AUTOCONF_GEN); \
	$(MV) .configure .configure- ;\
	echo "Moved .configure to .configure-"

distclean-recursive:
	for i in $(SUBDIRS); do ( cd $$i; $(MAKE) clean; $(RM) Makefile; ) done

tags:	force
	ctags -wtd *.cc *.h webcache/*.cc webcache/*.h dsdv/*.cc dsdv/*.h \
	dsr/*.cc dsr/*.h webcache/*.cc webcache/*.h lib/*.cc lib/*.h \
	../Tcl/*.cc ../Tcl/*.h 

TAGS:	force
	etags *.cc *.h webcache/*.cc webcache/*.h dsdv/*.cc dsdv/*.h \
	dsr/*.cc dsr/*.h webcache/*.cc webcache/*.h lib/*.cc lib/*.h \
	../Tcl/*.cc ../Tcl/*.h

tcl/lib/TAGS:	force
	( \
		cd tcl/lib; \
		$(TCLSH) ../../bin/tcl-expand.tcl ns-lib.tcl | grep '^### tcl-expand.tcl: 
		begin' | awk '{print $$5}' >.tcl_files; \
		etags --lang=none -r '/^[ \t]*proc[ \t]+\([^ \t]+\)/\1/' `cat .tcl_files`; \
		etags --append --lang=none -r '/^\([A-Z][^ \t]+\)[ \t]+\(instproc\|proc\)
		[ \t]+\([^ \t]+\)[ \t]+/\1::\3/' `cat .tcl_files`; \
	)

depend: $(SRC)
	$(MKDEP) $(CFLAGS) $(INCLUDES) $(SRC)

srctar:
	@cwd=`pwd` ; dir=`basename $$cwd` ; \
	    name=ns-`cat VERSION | tr A-Z a-z` ; \
	    tar=ns-src-`cat VERSION`.tar.gz ; \
	    list="" ; \
	    for i in `cat FILES` ; do list="$$list $$name/$$i" ; done; \
	    echo \
	    "(rm -f $$tar; cd .. ; ln -s $$dir $$name)" ; \
	     (rm -f $$tar; cd .. ; ln -s $$dir $$name) ; \
	    echo \
	    "(cd .. ; tar cfh $$tar [lots of files])" ; \
	     (cd .. ; tar cfh - $$list) | gzip -c > $$tar ; \
	    echo \
	    "rm ../$$name; chmod 444 $$tar" ;  \
	     rm ../$$name; chmod 444 $$tar

force:

test:	force
	./validate

.cpp.o:
	@rm -f $@
	$(CC) -c $(CFLAGS) $(INCLUDES) -o $@ $*.cpp


# Create makefile.vc for Win32 development by replacing:
# "# !include ..." 	-> 	"!include ..."
makefile.vc:	Makefile.in
	$(PERL) bin/gen-vcmake.pl < Makefile.in > makefile.vc
#	$(PERL) -pe 's/^# (\!include)/\!include/o' < Makefile.in > makefile.vc
\end{verbatim}
\normalsize

\subsection{The NS-2 Makefile for Linux}
\label{install:p2ps-ns2-build-linux}

At the beginning of section \ref{install:install}, instructions were given for
including Protolib and \agentj~ dependencies in the NS-2 Makefile for Linux
platforms.  A complete version of my Makefile, used to build NS-2 version 2.26
on a Linux box running a 2.6.7 kernel and version 1.5.0 of Sun's JDK, is
provided below:

\footnotesize
\begin{verbatim}
# Generated automatically from Makefile.in by configure.
#  Copyright (c) 1994, 1995, 1996
# 	The Regents of the University of California.  All rights reserved.
#
#  Redistribution and use in source and binary forms, with or without
#  modification, are permitted provided that: (1) source code distributions
#  retain the above copyright notice and this paragraph in its entirety, (2)
#  distributions including binary code include the above copyright notice and
#  this paragraph in its entirety in the documentation or other materials
#  provided with the distribution, and (3) all advertising materials mentioning
#  features or use of this software display the following acknowledgement:
#  ``This product includes software developed by the University of California,
#  Lawrence Berkeley Laboratory and its contributors.'' Neither the name of
#  the University nor the names of its contributors may be used to endorse
#  or promote products derived from this software without specific prior
#  written permission.
#  THIS SOFTWARE IS PROVIDED ``AS IS'' AND WITHOUT ANY EXPRESS OR IMPLIED
#  WARRANTIES, INCLUDING, WITHOUT LIMITATION, THE IMPLIED WARRANTIES OF
#  MERCHANTABILITY AND FITNESS FOR A PARTICULAR PURPOSE.
#
# @(#) $Header: 2002/10/09 15:34:11

#
# Various configurable paths (remember to edit Makefile.in, not Makefile)
#

# Top level hierarchy
prefix	= /usr/local
# Pathname of directory to install the binary
BINDEST	= /usr/local/bin
# Pathname of directory to install the man page
MANDEST	= /usr/local/man

BLANK	= # make a blank space.  DO NOT add anything to this line

# The following will be redefined under Windows (see WIN32 lable below)
#CC	= gcc
#CPP	= c++
CC	= gcc
CPP	= g++
LINK	= $(CPP)
MKDEP	= ./conf/mkdep
TCLSH	= /home/iandow/netsim/p2ps-ns2/ns-allinone-2.26/bin/tclsh8.3
TCL2C	= ../tclcl-1.0b13/tcl2c++
AR	= ar rc $(BLANK)

RANLIB	= ranlib
INSTALL	= /usr/bin/install -c
LN	= ln
TEST	= test
RM	= rm -f
MV      = mv
PERL	= /usr/bin/perl

# for diffusion
#DIFF_INCLUDES = "./diffusion3/main ./diffusion3/lib ./diffusion3/nr ./diffusion3/ns"

CCOPT	= 
STATIC	= 
LDFLAGS	= $(STATIC) -Wl,--rpath -Wl,$(JAVA_HOME)/jre/lib/i386/server/:$(JAVA_HOME)/jre/lib/i386/
LDOUT	= -o $(BLANK)

########################### Protolib Section #######################

PROTOLIB = ./protolib
PROTOLIB_INCLUDES = -I$(PROTOLIB)/common -I$(PROTOLIB)/ns
PROTOLIB_FLAGS = -DUNIX -DNS2 -DPROTO_DEBUG -DHAVE_ASSERT 

OBJ_PROTOLIB_CPP = protolib/ns/nsProtoAgent.o \
	protolib/common/protoSim.o \
	protolib/common/networkAddress.o \
	protolib/common/protocolTimer.o \
	protolib/common/debug.o

OBJ_AGENTJ_CPP = $(AGENTJ_UTILS)/LinkedList.o $(PAI_FACTORY)/PAIDispatcher.o \
	$(PAI_FACTORY)/PAIEngine.o $(PAI_FACTORY)/PAIFactory.o \
	$(PAI_FACTORY)/PAIMultipleListener.o $(PAI_FACTORY)/PAISocket.o \
	$(PAI_FACTORY)/PAITimer.o $(PAI_FACTORY)/PAIEnvironment.o \
	$(PAI_FACTORY)/PAIListener.o \
	$(PAI_FACTORY_NS)/PAINS2UDPSocket.o \
	$(PAI_FACTORY_NS)/PAINS2Timer.o \
	$(PAI_API)/PAI.o \
	$(PAI_AGENT)/PAIAgent.o $(PAI_AGENT)/PAISimpleAgent.o \
	$(AGENTJ_C_SRC)/C2JBroker.o $(AGENTJ_C_SRC)/Agentj.o \
	$(PAI_IMP_JNI)/JNIBridge.o $(PAI_IMP_JNI)/JNIImp.o



############################## AgentJ Section ########################

AGENTJ_SRC = $(AGENTJ)/src/c
AGENTJ_LIB_DIR = $(AGENTJ)/lib
AGENTJ_C_SRC = $(AGENTJ_SRC)/agentj
AGENTJ_UTILS = $(AGENTJ_SRC)/utils
PAI = $(AGENTJ_SRC)/pai
PAI_IMP = $(PAI)/imp
PAI_API = $(PAI)/api
PAI_AGENT = $(PAI_IMP)/agent
PAI_FACTORY = $(PAI_IMP)/factory
PAI_FACTORY_NET = $(PAI_FACTORY)/net
PAI_FACTORY_NS = $(PAI_FACTORY)/ns
PAI_IMP_JNI = $(PAI_IMP)/jni

AGENTJ_INCLUDES = -I$(JAVA_HOME)/include -I$(JAVA_HOME)/include/linux -I$(AGENTJ_C_SRC) -I$(AGENTJ_UTILS) -I$(PAI) -I$(PAI_AGENT) -I$(PAI_API) -I$(PAI_FACTORY) -I$(PAI_FACTORY_NET) -I$(PAI_FACTORY_NS) -I$(PAI_IMP_JNI) 
AGENTJ_LIB = -L$(JAVA_HOME)/jre/lib/i386/server/ -ljvm
AGENTJ_SHARED_LDFLAGS = -shared

DEFINE	= -DTCP_DELAY_BIND_ALL -DNO_TK -DTCLCL_CLASSINSTVAR  -DNDEBUG -DLINUX_TCP_HEADER -DUSE_SHM -DHAVE_LIBTCLCL -DHAVE_TCLCL_H -DHAVE_LIBOTCL1_0A8 -DHAVE_OTCL_H -DHAVE_LIBTK8_3 -DHAVE_TK_H -DHAVE_LIBTCL8_3 -DHAVE_TCL_H  -DHAVE_CONFIG_H -DNS_DIFFUSION -DSMAC_NO_SYNC -DSTL_NAMESPACE=std -DUSE_SINGLE_ADDRESS_SPACE

INCLUDES = \
	$(PROTOLIB_INCLUDES) \
	$(AGENTJ_INCLUDES) \
	-I.  \
	-I/home/iandow/netsim/p2ps-ns2/ns-allinone-2.26/tclcl-1.0b13 -I/home/iandow/netsim/p2ps-ns2/ns-allinone-2.26/otcl-1.0a8 -I/home/iandow/netsim/p2ps-ns2/ns-allinone-2.26/include -I/home/iandow/netsim/p2ps-ns2/ns-allinone-2.26/include -I/usr/include/pcap \
	-I/home/iandow/netsim/p2ps-ns2/ns-allinone-2.26/ns-2.26/common \
	-I/home/iandow/netsim/p2ps-ns2/ns-allinone-2.26/tclcl-1.0b13 \
	-I/home/iandow/netsim/p2ps-ns2/ns-allinone-2.26/otcl-1.0a8 \
	-I/home/iandow/netsim/p2ps-ns2/ns-allinone-2.26/include \
	-I/home/iandow/netsim/p2ps-ns2/ns-allinone-2.26/include \
	-I/usr/include/pcap \
	-I./tcp -I./common -I./link -I./queue \
	-I./adc -I./apps -I./mac -I./mobile -I./trace \
	-I./routing -I./tools -I./classifier -I./mcast \
	-I./diffusion3/lib/main -I./diffusion3/lib \
	-I./diffusion3/lib/nr -I./diffusion3/ns \
	-I./diffusion3/diffusion -I./asim/ -I./qs

LIB	= \
	-L/home/iandow/netsim/p2ps-ns2/ns-allinone-2.26/tclcl-1.0b13 -ltclcl -L/home/iandow/netsim/p2ps-ns2/ns-allinone-2.26/otcl-1.0a8 -lotcl -L/home/iandow/netsim/p2ps-ns2/ns-allinone-2.26/lib -ltk8.3 -L/home/iandow/netsim/p2ps-ns2/ns-allinone-2.26/lib -ltcl8.3 \
	-L/usr/X11R6/lib -lXext -lX11 \
	 -lnsl -lpcap -ldl \
	$(AGENTJ_LIB) \
	-lm 
#	-L${exec_prefix}/lib \

CFLAGS	= -g $(CCOPT) $(DEFINE) $(PROTOLIB_FLAGS)

# Explicitly define compilation rules since SunOS 4's make doesn't like gcc.
# Also, gcc does not remove the .o before forking 'as', which can be a
# problem if you don't own the file but can write to the directory.
.SUFFIXES: .cc	# $(.SUFFIXES)

.cc.o:
	@rm -f $@
	$(CPP) -c $(CFLAGS) $(INCLUDES) -o $@ $*.cc

.c.o:
	@rm -f $@
	$(CC) -c $(CFLAGS) $(INCLUDES) -o $@ $*.c

.cpp.o: @rm -f $@ 
	$(CC) -c $(CFLAGS) $(INCLUDES) -o $@ $*.cpp

GEN_DIR	= gen/
LIB_DIR	= lib/
NS	= ns
NSX	= nsx
NSE	= nse

# To allow conf/makefile.win overwrite this macro
# We will set these two macros to empty in conf/makefile.win since VC6.0
# does not seem to support the STL in gcc 2.8 and up. 
OBJ_STL = diffusion3/lib/nr/nr.o diffusion3/lib/dr.o \
	diffusion3/ns/diffagent.o diffusion3/ns/diffrtg.o \
	diffusion3/ns/difftimer.o \
	diffusion3/diffusion/diffusion.o \
	diffusion3/lib/main/attrs.o \
	diffusion3/lib/main/iodev.o \
	diffusion3/lib/main/timers.o \
	diffusion3/lib/main/events.o \
	diffusion3/lib/main/message.o \
	diffusion3/lib/main/stats.o \
	diffusion3/lib/main/tools.o \
	diffusion3/lib/drivers/rpc_stats.o \
	diffusion3/apps/sysfilters/gradient.o \
	diffusion3/apps/sysfilters/log.o \
	diffusion3/apps/sysfilters/tag.o \
	diffusion3/apps/sysfilters/srcrt.o \
	diffusion3/lib/diffapp.o \
	diffusion3/apps/pingapp/ping_sender.o \
	diffusion3/apps/pingapp/ping_receiver.o \
	diffusion3/apps/pingapp/ping_common.o \
	diffusion3/apps/pingapp/push_receiver.o \
	diffusion3/apps/pingapp/push_sender.o \
	diffusion3/apps/gear/geo-attr.o \
	diffusion3/apps/gear/geo-routing.o \
	diffusion3/apps/gear/geo-tools.o \
	nix/hdr_nv.o nix/classifier-nix.o \
	nix/nixnode.o nix/nixvec.o \
	nix/nixroute.o

NS_TCL_LIB_STL = tcl/lib/ns-diffusion.tcl 



# WIN32: uncomment the following line to include specific make for VC++
# !include <conf/makefile.win>

OBJ_CC = \
	tools/random.o tools/rng.o tools/ranvar.o common/misc.o common/timer-handler.o \
	common/scheduler.o common/object.o common/packet.o \
	common/ip.o routing/route.o common/connector.o common/ttl.o \
	trace/trace.o trace/trace-ip.o \
	classifier/classifier.o classifier/classifier-addr.o \
	classifier/classifier-hash.o \
	classifier/classifier-virtual.o \
	classifier/classifier-mcast.o \
	classifier/classifier-bst.o \
	classifier/classifier-mpath.o mcast/replicator.o \
	classifier/classifier-mac.o \
	classifier/classifier-qs.o \
	classifier/classifier-port.o src_rtg/classifier-sr.o \
        src_rtg/sragent.o src_rtg/hdr_src.o adc/ump.o \
	qs/qsagent.o qs/hdr_qs.o \
	apps/app.o apps/telnet.o tcp/tcplib-telnet.o \
	tools/trafgen.o trace/traffictrace.o tools/pareto.o \
	tools/expoo.o tools/cbr_traffic.o \
	adc/tbf.o adc/resv.o adc/sa.o tcp/saack.o \
	tools/measuremod.o adc/estimator.o adc/adc.o adc/ms-adc.o \
	adc/timewindow-est.o adc/acto-adc.o \
        adc/pointsample-est.o adc/salink.o adc/actp-adc.o \
	adc/hb-adc.o adc/expavg-est.o\
	adc/param-adc.o adc/null-estimator.o \
	adc/adaptive-receiver.o apps/vatrcvr.o adc/consrcvr.o \
	common/agent.o common/message.o apps/udp.o \
	common/session-rtp.o apps/rtp.o tcp/rtcp.o \
	common/ivs.o \
	tcp/tcp.o tcp/tcp-sink.o tcp/tcp-reno.o \
	tcp/tcp-newreno.o \
	tcp/tcp-vegas.o tcp/tcp-rbp.o tcp/tcp-full.o tcp/rq.o \
	baytcp/tcp-full-bay.o baytcp/ftpc.o baytcp/ftps.o \
	tcp/scoreboard.o tcp/scoreboard-rq.o tcp/tcp-sack1.o tcp/tcp-fack.o \
	tcp/tcp-asym.o tcp/tcp-asym-sink.o tcp/tcp-fs.o \
	tcp/tcp-asym-fs.o tcp/tcp-qs.o \
	tcp/tcp-int.o tcp/chost.o tcp/tcp-session.o \
	tcp/nilist.o \
	tools/integrator.o tools/queue-monitor.o \
	tools/flowmon.o tools/loss-monitor.o \
	queue/queue.o queue/drop-tail.o \
	adc/simple-intserv-sched.o queue/red.o \
	queue/semantic-packetqueue.o queue/semantic-red.o \
	tcp/ack-recons.o \
	queue/sfq.o queue/fq.o queue/drr.o queue/srr.o queue/cbq.o \
	queue/jobs.o queue/marker.o queue/demarker.o \
	link/hackloss.o queue/errmodel.o queue/fec.o\
	link/delay.o tcp/snoop.o \
	gaf/gaf.o \
	link/dynalink.o routing/rtProtoDV.o common/net-interface.o \
	mcast/ctrMcast.o mcast/mcast_ctrl.o mcast/srm.o \
	common/sessionhelper.o queue/delaymodel.o \
	mcast/srm-ssm.o mcast/srm-topo.o \
	apps/mftp.o apps/mftp_snd.o apps/mftp_rcv.o \
	apps/codeword.o \
	routing/alloc-address.o routing/address.o \
	$(LIB_DIR)int.Vec.o $(LIB_DIR)int.RVec.o \
	$(LIB_DIR)dmalloc_support.o \
	webcache/http.o webcache/tcp-simple.o webcache/pagepool.o \
	webcache/inval-agent.o webcache/tcpapp.o webcache/http-aux.o \
	webcache/mcache.o webcache/webtraf.o \
	webcache/webserver.o \
	webcache/logweb.o \
	empweb/empweb.o \
	empweb/empftp.o \
	realaudio/realaudio.o \
	mac/lanRouter.o classifier/filter.o \
	common/pkt-counter.o \
	common/Decapsulator.o common/Encapsulator.o \
	common/encap.o \
	mac/channel.o mac/mac.o mac/ll.o mac/mac-802_11.o \
	mac/mac-802_3.o mac/mac-tdma.o mac/smac.o \
	mobile/mip.o mobile/mip-reg.o mobile/gridkeeper.o \
	mobile/propagation.o mobile/tworayground.o \
	mobile/antenna.o mobile/omni-antenna.o \
	mobile/shadowing.o mobile/shadowing-vis.o mobile/dumb-agent.o \
	common/bi-connector.o common/node.o \
	common/mobilenode.o \
	mac/arp.o mobile/god.o mobile/dem.o \
	mobile/topography.o mobile/modulation.o \
	queue/priqueue.o queue/dsr-priqueue.o \
	mac/phy.o mac/wired-phy.o mac/wireless-phy.o \
	mac/mac-timers.o trace/cmu-trace.o mac/varp.o \
	dsdv/dsdv.o dsdv/rtable.o queue/rtqueue.o \
	routing/rttable.o \
	imep/imep.o imep/dest_queue.o imep/imep_api.o \
	imep/imep_rt.o imep/rxmit_queue.o imep/imep_timers.o \
	imep/imep_util.o imep/imep_io.o \
	tora/tora.o tora/tora_api.o tora/tora_dest.o \
	tora/tora_io.o tora/tora_logs.o tora/tora_neighbor.o \
	dsr/dsragent.o dsr/hdr_sr.o dsr/mobicache.o dsr/path.o \
	dsr/requesttable.o dsr/routecache.o dsr/add_sr.o \
	dsr/dsr_proto.o dsr/flowstruct.o dsr/linkcache.o \
	dsr/simplecache.o dsr/sr_forwarder.o \
	aodv/aodv_logs.o aodv/aodv.o \
	aodv/aodv_rtable.o aodv/aodv_rqueue.o \
	common/ns-process.o \
	satellite/satgeometry.o satellite/sathandoff.o \
	satellite/satlink.o satellite/satnode.o \
	satellite/satposition.o satellite/satroute.o \
	satellite/sattrace.o \
	rap/raplist.o rap/rap.o rap/media-app.o rap/utilities.o \
	common/fsm.o tcp/tcp-abs.o \
	diffusion/diffusion.o diffusion/diff_rate.o diffusion/diff_prob.o \
	diffusion/diff_sink.o diffusion/flooding.o diffusion/omni_mcast.o \
	diffusion/hash_table.o diffusion/routing_table.o diffusion/iflist.o \
	tcp/tfrc.o tcp/tfrc-sink.o mobile/energy-model.o apps/ping.o tcp/tcp-rfc793edu.o \
	queue/rio.o queue/semantic-rio.o tcp/tcp-sack-rh.o tcp/scoreboard-rh.o \
	plm/loss-monitor-plm.o plm/cbr-traffic-PP.o \
	linkstate/hdr-ls.o \
	mpls/classifier-addr-mpls.o mpls/ldp.o mpls/mpls-module.o \
	routing/rtmodule.o classifier/classifier-hier.o \
	routing/addr-params.o \
	routealgo/rnode.o \
	routealgo/bfs.o \
	routealgo/rbitmap.o \
	routealgo/rlookup.o \
	routealgo/routealgo.o \
	diffserv/dsred.o diffserv/dsredq.o \
	diffserv/dsEdge.o diffserv/dsCore.o \
	diffserv/dsPolicy.o diffserv/ew.o\
	queue/red-pd.o queue/pi.o queue/vq.o queue/rem.o \
	queue/gk.o \
	pushback/rate-limit.o pushback/rate-limit-strategy.o \
	pushback/ident-tree.o pushback/agg-spec.o \
	pushback/logging-data-struct.o \
	pushback/rate-estimator.o \
	pushback/pushback-queue.o pushback/pushback.o \
	common/parentnode.o trace/basetrace.o \
	common/simulator.o asim/asim.o \
	common/scheduler-map.o common/splay-scheduler.o \
	linkstate/ls.o linkstate/rtProtoLS.o \
	pgm/classifier-pgm.o pgm/pgm-agent.o pgm/pgm-sender.o \
	pgm/pgm-receiver.o mcast/rcvbuf.o \
	mcast/classifier-lms.o mcast/lms-agent.o mcast/lms-receiver.o \
	mcast/lms-sender.o \
	$(OBJ_STL)


# don't allow comments to follow continuation lines

#  mac-csma.o mac-multihop.o\
#	sensor-nets/landmark.o mac-simple-wireless.o \
#	sensor-nets/tags.o sensor-nets/sensor-query.o \
#	sensor-nets/flood-agent.o \

# what was here before is now in emulate/
OBJ_C =

OBJ_COMPAT = $(OBJ_GETOPT) common/win32.o
#XXX compat/win32x.o compat/tkConsole.o

OBJ_EMULATE_CC = \
	emulate/net-ip.o \
	emulate/net.o \
	emulate/tap.o \
	emulate/ether.o \
	emulate/internet.o \
	emulate/ping_responder.o \
	emulate/arp.o \
	emulate/icmp.o \
	emulate/net-pcap.o \
	emulate/nat.o  \
	emulate/iptap.o \
	emulate/tcptap.o

OBJ_EMULATE_C = \
	emulate/inet.o

OBJ_GEN = $(GEN_DIR)version.o $(GEN_DIR)ns_tcl.o $(GEN_DIR)ptypes.o

SRC =	$(OBJ_C:.o=.c) $(OBJ_CC:.o=.cc) \
	$(OBJ_EMULATE_C:.o=.c) $(OBJ_EMULATE_CC:.o=.cc) \
	common/tclAppInit.cc common/tkAppInit.cc $(OBJ_CPP:.o=.cpp)

OBJ =	$(OBJ_C) $(OBJ_CC) $(OBJ_GEN) $(OBJ_COMPAT) $(OBJ_PROTOLIB_CPP) $(OBJ_AGENTJ_CPP)

CLEANFILES = ns nse nsx ns.dyn $(OBJ) $(OBJ_EMULATE_CC) \
	$(OBJ_EMULATE_C) common/tclAppInit.o \
	$(GEN_DIR)* $(NS).core core core.$(NS) core.$(NSX) core.$(NSE) \
	common/ptypes2tcl common/ptypes2tcl.o 

SUBDIRS=\
	indep-utils/cmu-scen-gen/setdest \
	indep-utils/webtrace-conv/dec \
	indep-utils/webtrace-conv/epa \
	indep-utils/webtrace-conv/nlanr \
	indep-utils/webtrace-conv/ucb

BUILD_NSE = nse

all: $(NS) $(BUILD_NSE) all-recursive


all-recursive:
	for i in $(SUBDIRS); do ( cd $$i; $(MAKE) all; ) done

$(NS): $(OBJ) common/tclAppInit.o Makefile
	$(LINK) $(LDFLAGS) $(LDOUT)$@ \
		common/tclAppInit.o $(OBJ) $(LIB)

Makefile: Makefile.in
	@echo "Makefile.in is newer than Makefile."
	@echo "You need to re-run configure."
	false

$(NSE): $(OBJ) common/tclAppInit.o $(OBJ_EMULATE_CC) $(OBJ_EMULATE_C)
	$(LINK) $(LDFLAGS) $(LDOUT)$@ \
		common/tclAppInit.o $(OBJ) \
		$(OBJ_EMULATE_CC) $(OBJ_EMULATE_C)  $(LIB) 

ns.dyn: $(OBJ) common/tclAppInit.o
	$(LINK) $(LDFLAGS) -o $@ \
		common/tclAppInit.o $(OBJ) $(LIB)

PURIFY	= purify -cache-dir=/tmp
ns-pure: $(OBJ) common/tclAppInit.o
	$(PURIFY) $(LINK) $(LDFLAGS) -o $@ \
		common/tclAppInit.o $(OBJ) $(LIB)

NS_TCL_LIB = \
	tcl/lib/ns-compat.tcl \
	tcl/lib/ns-default.tcl \
	tcl/lib/ns-errmodel.tcl \
	tcl/lib/ns-lib.tcl \
	tcl/lib/ns-link.tcl \
	tcl/lib/ns-mobilenode.tcl \
	tcl/lib/ns-sat.tcl \
	tcl/lib/ns-cmutrace.tcl \
	tcl/lib/ns-node.tcl \
	tcl/lib/ns-rtmodule.tcl \
	tcl/lib/ns-hiernode.tcl \
	tcl/lib/ns-packet.tcl \
	tcl/lib/ns-queue.tcl \
	tcl/lib/ns-source.tcl \
	tcl/lib/ns-nam.tcl \
	tcl/lib/ns-trace.tcl \
	tcl/lib/ns-agent.tcl \
	tcl/lib/ns-random.tcl \
	tcl/lib/ns-namsupp.tcl \
	tcl/lib/ns-address.tcl \
	tcl/lib/ns-intserv.tcl \
	tcl/lib/ns-autoconf.tcl \
	tcl/rtp/session-rtp.tcl \
	tcl/lib/ns-mip.tcl \
	tcl/rtglib/dynamics.tcl \
	tcl/rtglib/route-proto.tcl \
	tcl/rtglib/algo-route-proto.tcl \
	tcl/rtglib/ns-rtProtoLS.tcl \
        tcl/interface/ns-iface.tcl \
	tcl/mcast/BST.tcl \
        tcl/mcast/ns-mcast.tcl \
        tcl/mcast/McastProto.tcl \
        tcl/mcast/DM.tcl \
	tcl/mcast/srm.tcl \
	tcl/mcast/srm-adaptive.tcl \
	tcl/mcast/srm-ssm.tcl \
	tcl/mcast/timer.tcl \
	tcl/mcast/McastMonitor.tcl \
	tcl/mcast/mftp_snd.tcl \
	tcl/mcast/mftp_rcv.tcl \
	tcl/mcast/mftp_rcv_stat.tcl \
	tcl/mobility/dsdv.tcl \
	tcl/mobility/dsr.tcl \
        tcl/ctr-mcast/CtrMcast.tcl \
        tcl/ctr-mcast/CtrMcastComp.tcl \
        tcl/ctr-mcast/CtrRPComp.tcl \
	tcl/rlm/rlm.tcl \
	tcl/rlm/rlm-ns.tcl \
	tcl/session/session.tcl \
	tcl/lib/ns-route.tcl \
	tcl/emulate/ns-emulate.tcl \
	tcl/lan/vlan.tcl \
	tcl/lan/abslan.tcl \
	tcl/lan/ns-ll.tcl \
	tcl/lan/ns-mac.tcl \
	tcl/webcache/http-agent.tcl \
	tcl/webcache/http-server.tcl \
	tcl/webcache/http-cache.tcl \
	tcl/webcache/http-mcache.tcl \
	tcl/webcache/webtraf.tcl \
	tcl/webcache/empweb.tcl \
	tcl/webcache/empftp.tcl \
	tcl/plm/plm.tcl \
	tcl/plm/plm-ns.tcl \
	tcl/plm/plm-topo.tcl \
	tcl/mpls/ns-mpls-classifier.tcl \
	tcl/mpls/ns-mpls-ldpagent.tcl \
	tcl/mpls/ns-mpls-node.tcl \
	tcl/mpls/ns-mpls-simulator.tcl \
	tcl/lib/ns-pushback.tcl \
	tcl/lib/ns-srcrt.tcl \
	tcl/mcast/ns-lms.tcl \
	tcl/lib/ns-qsnode.tcl \
	$(NS_TCL_LIB_STL)

$(GEN_DIR)ns_tcl.cc: $(NS_TCL_LIB)
	$(TCLSH) bin/tcl-expand.tcl tcl/lib/ns-lib.tcl $(NS_TCL_LIB_STL) | $(TCL2C) et_ns_lib > $@

$(GEN_DIR)version.c: VERSION
	$(RM) $@
	$(TCLSH) bin/string2c.tcl version_string < VERSION > $@

$(GEN_DIR)ptypes.cc: common/ptypes2tcl common/packet.h
	./common/ptypes2tcl > $@

common/ptypes2tcl: common/ptypes2tcl.o
	$(LINK) $(LDFLAGS) $(LDOUT)$@ common/ptypes2tcl.o

common/ptypes2tcl.o: common/ptypes2tcl.cc common/packet.h

libagentj.so: $(OBJ) common/tclAppInit.o
	$(LINK) $(AGENTJ_SHARED_LDFLAGS) -o $@ common/tclAppInit.o $(OBJ) $(LIB)
	mv libagentj.so $(AGENTJ_LIB_DIR)

install: force install-ns install-man install-recursive

install-ns: force
	$(INSTALL) -m 555 -o bin -g bin ns $(DESTDIR)$(BINDEST)

install-man: force
	$(INSTALL) -m 444 -o bin -g bin ns.1 $(DESTDIR)$(MANDEST)/man1

install-recursive: force
	for i in $(SUBDIRS); do ( cd $$i; $(MAKE) install; ) done

clean:
	$(RM) $(CLEANFILES)

AUTOCONF_GEN = tcl/lib/ns-autoconf.tcl
distclean: distclean-recursive
	$(RM) $(CLEANFILES) Makefile config.cache config.log config.status \
	    autoconf.h gnuc.h os-proto.h $(AUTOCONF_GEN); \
	$(MV) .configure .configure- ;\
	echo "Moved .configure to .configure-"

distclean-recursive:
	for i in $(SUBDIRS); do ( cd $$i; $(MAKE) clean; $(RM) Makefile; ) done

tags:	force
	ctags -wtd *.cc *.h webcache/*.cc webcache/*.h dsdv/*.cc dsdv/*.h \
	dsr/*.cc dsr/*.h webcache/*.cc webcache/*.h lib/*.cc lib/*.h \
	../Tcl/*.cc ../Tcl/*.h 

TAGS:	force
	etags *.cc *.h webcache/*.cc webcache/*.h dsdv/*.cc dsdv/*.h \
	dsr/*.cc dsr/*.h webcache/*.cc webcache/*.h lib/*.cc lib/*.h \
	../Tcl/*.cc ../Tcl/*.h

tcl/lib/TAGS:	force
	( \
		cd tcl/lib; \
		$(TCLSH) ../../bin/tcl-expand.tcl ns-lib.tcl | grep '^### tcl-expand.tcl: begin' | awk '{print $$5}' >.tcl_files; \
		etags --lang=none -r '/^[ \t]*proc[ \t]+\([^ \t]+\)/\1/' `cat .tcl_files`; \
		etags --append --lang=none -r '/^\([A-Z][^ \t]+\)[ \t]+\(instproc\|proc\)[ \t]+\([^ \t]+\)[ \t]+/\1::\3/' `cat .tcl_files`; \
	)

depend: $(SRC)
	$(MKDEP) $(CFLAGS) $(INCLUDES) $(SRC)

srctar:
	@cwd=`pwd` ; dir=`basename $$cwd` ; \
	    name=ns-`cat VERSION | tr A-Z a-z` ; \
	    tar=ns-src-`cat VERSION`.tar.gz ; \
	    list="" ; \
	    for i in `cat FILES` ; do list="$$list $$name/$$i" ; done; \
	    echo \
	    "(rm -f $$tar; cd .. ; ln -s $$dir $$name)" ; \
	     (rm -f $$tar; cd .. ; ln -s $$dir $$name) ; \
	    echo \
	    "(cd .. ; tar cfh $$tar [lots of files])" ; \
	     (cd .. ; tar cfh - $$list) | gzip -c > $$tar ; \
	    echo \
	    "rm ../$$name; chmod 444 $$tar" ;  \
	     rm ../$$name; chmod 444 $$tar

force:

test:	force
	./validate

# Create makefile.vc for Win32 development by replacing:
# "# !include ..." 	-> 	"!include ..."
makefile.vc:	Makefile.in
	$(PERL) bin/gen-vcmake.pl < Makefile.in > makefile.vc
#	$(PERL) -pe 's/^# (\!include)/\!include/o' < Makefile.in > makefile.vc
\end{verbatim}
\normalsize


\subsection{What's included in the \agentj~ Release?}
\label{install:p2ps-ns2-included}

The \agentj distribution consists of several co-operating
software stacks, which are described in the following chapters of the 
manual.  It includes the PAI interface to Protolib (described in Chapt.
\ref{pai} and the Java NS2 agent extensions, described in Chapt.
\ref{jni}. \agentj also has an accompanying package, called P2PSX,
which provides a P2P framework within NS2 \cite{p2psx}.

\section{Configuration}
\label{install:configuration}

\agentj has fixed a number of configuration issues for this release. \agentj
now uses the standard environment variables to configure the location of the
agentj shared library (for JNI) and for the Java classpath required by the
package.  There are a couple of points, however: 

\begin{itemize}

\item \textbf{JNI Configuration:} On my apple Mac, dynamic libraries for JNI
are named specifically for this purpose, using the format
lib$<$libname$>$.jnilib.  On Linux, the format is lib$<$libname$>$.so.  This
type of format is required, otherwise, the Java implementation cannot use the
LD\_LIBRARY\_PATH environment variable to find the shard library (we use this
in this release of \agentj). For ports to other platforms you will have to use
the format required for that platform and include the necessary libraries
required by the JNI interface.  This is normally straight forward and
information can be found in the Java Tutorial \cite{javaTutorial} for settings
for the various platforms.  

\item \textbf{Classpath:} The CLASSPATH environment variable is used to
initialise the JVM that is created by \agentj. Therefore, ANY paths required by
your Java application running within NS2 will need to be specified using this
environment variable.  Since, we dynamically create a JVM from within the C++
code, this is really the only reliable mechanism for setting the CLASSPATH
within the JVM.   Normally, one would tend to specify these on the command line
dynamically but this is not possible with \agentj.  Further, if you build your
own \agentj~ objects, then the classpaths for these will need to be included. 

\end{itemize}

\section{\agentj Logging}
\label{install:logging}

The AGENTJDEBUG environment variable is a course-grained mechanim which can be
used to turn logging on or off. However, if turned on, log4j can be configured
in a number of ways. \agentj uses a simple Java package, called Autolog, which
is used to extend the log4j discovery mechanisms, allowing the user to specify
local log4j configuration files, outside the scope of the classpath.  This is
described in more detail in the next two sections.  

\subsection{AutoLog Overview}
\label{whatis}

AutoLog is a simple interface to initialise the Log4j logging system, which
extends the discovery mechanisms for XML configuration files and
provides an auto configuration mode when there is no default configuration 
chosen by the user.  In short, Autolog always tries to make the best
of any particular environment. 


The log4j library does not make any assumptions about its 
environment. In particular, there are no default log4j appenders,
which results in an Error like the following:

\footnotesize
\begin{verbatim}
log4j:WARN No appenders could be found for logger (MyApp).
log4j:WARN Please initialize the log4j system properly.
Process terminated with exit code 0
\end{verbatim}
\normalsize

\noindent then, thereafter, no logging message are output.  AutoLog
tries to configure the loggin system under these circumstances to
use the PatternLayout and to set the logging to the WARN level.
However, it does more than this. The next section describes the
discovery procedure.


\subsection{AutoLog Discovery}
\label{whatis}

The following procedure takes place:

\begin{itemize}
\item \textbf{Log4J:} \sloppypar The default log4j initialization procedure is attempt.  This
searches the CLASSPATH for configuration files which you
set using the Java property ``log4j.configuration''.  You can set
this at the command-line using something like:
\footnotesize
\begin{verbatim}
java -Dlog4j.configuration=myfile.xml MyClass
\end{verbatim}
\normalsize

Note that the file you specify MUST be located somewhere within your CLASSPATH. 
\item \textbf{Autolog:} The autolog Java property is tried.  This allows you
to specify an absolute path/URL to your filename containing your XML 
configuration.  This property is called ``log4j.configuration.file''. You can
set this using something like this:
\footnotesize
\begin{verbatim}
java -Dlog4j.configuration.file=myfile.xml MyClass
\end{verbatim}
\normalsize

\item \textbf{Default:} If the other two fail, then we resort to apply a default
appender to all loggers created within your code, which uses the
Pattern layout using the following format:

\footnotesize
\begin{verbatim}
%-7p: %l%nMESSAGE: %m (%d)%n%n
\end{verbatim}
\normalsize

\noindent which results in your logging statement being output in the following fomat


\footnotesize
\begin{verbatim}
WARN   : examples.SimpleLogging.<init>(SimpleLogging.java:19)
MESSAGE: Here is some WARN (2004-06-30 11:42:27,037)

FATAL  : examples.SimpleLogging.<init>(SimpleLogging.java:20)
MESSAGE: Here is some FATAL (2004-06-30 11:42:27,038)
\end{verbatim}
\normalsize
\end{itemize}

\subsection{Example XML Configuration}
\label{example}

In the autolog config directory, there are a number of example scripts.  Here,
we show an example script that gives the same output as the default
logging format described above, for comparison:

\footnotesize
\begin{verbatim}
<?xml version="1.0" encoding="UTF-8" ?>
<!DOCTYPE log4j:configuration SYSTEM "log4j.dtd">

<log4j:configuration xmlns:log4j='http://jakarta.apache.org/log4j/'>
        <appender name="ToTheScreen" class="org.apache.log4j.ConsoleAppender">
           <layout class="org.apache.log4j.PatternLayout">
              <param name="ConversionPattern" 
                     value="%-7p: %l%nMESSAGE: %m  (%d)%n %n"/>
	   </layout>
	</appender>
 
        <category name="A0123456789">  	
	   <priority value ="info" />
        </category>  	

	<root>
	   <priority value ="debug" />
   	   <appender-ref ref="ToTheScreen" />
	</root>
	
</log4j:configuration> 
\end{verbatim}
\normalsize

\section{Conclusion}

This chapter described the installation of the core packages needed
for P2PS-NS2.  The Protolib library needs to be installed first, followed by
the P2PS-NS2 installation.  Both installation require the editing
of the NS Makefile.in file in order to add the correct dependencies into
NS2.  P2PS-NS requires the installation of both the static and shared
libraries for the NS2 executable and the JNI bindings describes later
in this manual.
 
 
 
