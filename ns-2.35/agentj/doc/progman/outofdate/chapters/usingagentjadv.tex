\chapter{Advanced \agentj}
\label{jnipai}

\section{\agentj and PAI}

In the last section, we discussed the way Java objects could be
attached to Java agents and invoke from within NS2 simulations.
In this chapter, an overview of how such Java nodes can be used 
to send packages between NS2 nodes by using the PAI interface,
described in Chapt. \ref{pai}.  The Java interface contains a
an interface to PAI through JNI that enables the Java objects
to create sockets, attach listeners to the sockets and trigger 
events.

\subsection{Using the Java PAI Interface in Ns2 Java Objects}
\label{jnipai:jpaireqs}
 
Each Java objects that has been attached to an NS2 node
must implement the \emph{PAIAccessInterface} given
below, which can be found in the pai.broker package within the
source tree:

\footnotesize
\begin{verbatim}
package pai.broker;

import pai.api.PAIInterface;

public interface PAIAccessInterface {

    public void setPAI(PAIInterface pai);
}

\end{verbatim}
\normalsize


\emph{PAIAccessInterface} provides a mechanism for the 
\emph{JavaBroker} object to create a \emph{PAIInterface} 
object to the JNI PAI implementation and pass this reference 
to your Java code.  You can then use this reference 
directly to make PAI calls just as you would if you were 
using PAI directly.

This mechanism managers the creation and deletion of the
PAI JNI implementation and sets variables in the JNI before
each invocation so that it has the correct reference to the 
object it is dealing with at that moment.  Briefly, the 
\emph{JavaBeker} only create \textbf{one} instance of
the PAI JNI implementation. This means that before each
call it must set the reference to the actual NS2 node it is
about to issue a command to enabling the interface to 
create the appropriate binding to PAI at the lower levels.
This design adds a small overhead to each call but saves
a substantial amount of memory since it efficiently uses
one instance of the code rather that one for each node,
which would increase memory consumption greatly (i.e.
image if you had thousands of nodes).


\section{Example 1: Sending Data From One Node to Another}
\label{jni:node2node}

This example uses the Java \emph{PAICommands} class to 
send data between two NS2 nodes.  The actual Java
code specifies which nodes to communicate with.
This simple example demonstrates how Java objects
can be attached to an NS2 nodes and used to 
create sockets and send data between nodes.

\subsection{The TCL Side}
\label{jni:tclside}

The following is the TCL script part of the implementation, which creates two 
\emph{JavaAgent} NS2 nodes attaches the PAICommands Java object
to them, initializes them and then sends data from the first node to
the second by setting the NS 2 address of the second node directly from
the script, using the \emph{setSendTo} command.

\footnotesize
\begin{verbatim}
# Create multicast enabled simulator instance
set ns_ [new Simulator]

# Create two nodes
set n1 [$ns_ node]
set n2 [$ns_ node]

# Put a link between them
$ns_ duplex-link $n1 $n2 64kb 100ms DropTail
$ns_ queue-limit $n1 $n2 100
$ns_ duplex-link-op $n1 $n2 queuePos 0.5
$ns_ duplex-link-op $n1 $n2 orient right

puts "Creating JavaAgent NS2 agents and attach them to the nodes..."   
set p1 [new Agent/JavaAgent]
$ns_ attach-agent $n1 $p1

set p2 [new Agent/JavaAgent]
$ns_ attach-agent $n2 $p2

puts "In script: Initializing agents  ..." 
	
$ns_ at 0.0 "$p1 initAgent"
$ns_ at 0.0 "$p2 initAgent"

puts "Setting Java Object to use by each agent ..." 

$ns_ at 0.0 "$p1 setClass 
/Users/scmijt/Apps/nrl/p2ps-ns2/classes pai.examples.ns2.PAICommands"
$ns_ at 0.0 "$p2 setClass 
/Users/scmijt/Apps/nrl/p2ps-ns2/classes pai.examples.ns2.PAICommands"

puts "Starting simulation ..." 

$ns_ at 0.0 "$p1 javaCommand init"
$ns_ at 0.0 "$p2 javaCommand init"

$ns_ at 0.0 "$p1 javaCommand setSendTo [$n2 node-addr]"
$ns_ at 0.0 "$p1 javaCommand start"

$ns_ at 10.0 "finish $ns_"

proc finish {ns_} {
$ns_ halt
delete $ns_
}

$ns_ run

\end{verbatim}
\normalsize

The Java classes are located and instantiate as described in Sect.
\ref{jni:basic}. Now, we can use the \emph{javaCommand init} instruction 
to initialize the Java  nodes, set the send to address (the node 
where the data will be sent to) and start the node off, which 
results in it sending the data.

\subsection{The Java Side}
\label{jni:javaside}

On the java side \emph{PAICommands.java} implements various
instructions to help send data and to trigger timers etc:

\footnotesize
\begin{verbatim}
package pai.examples.ns2;

import pai.broker.CommandInterface;
import pai.broker.PAIAccessInterface;
import pai.api.PAIInterface;
import pai.net.DatagramSocket;
import pai.net.DatagramPacket;
import pai.net.InetAddress;
import pai.impl.PAITimer;
import pai.impl.Logging;
import pai.event.PAISocketEvent;
import pai.event.PAISocketListener;
import java.net.SocketException;
import java.io.IOException;

public class PAICommands implements CommandInterface, PAIAccessInterface,
                                        PAISocketListener {
    PAIInterface pai;
    String sendTo;
    DatagramSocket s;
    PAITimer t;
    int count=0;

    public void init() {
        try {
            s = pai.addSocket(5555);
            pai.addPAISocketListener(s,this);
        } catch (SocketException e) {
            System.out.println("Error opening socket");
        }
        catch (IOException ep) {
            System.out.println("Error opening socket");
        }
    }

    void start() {
	    timerTriggered();  // transmit first packet right away
        }

    public void dataReceived(PAISocketEvent sv) {
        try {
            ++count;
            byte b[] = new byte[15];
            DatagramPacket p = new DatagramPacket(b,b.length);
            pai.receive(s, p);
            if (Logging.isEnabled()) {
                System.out.println("PAICommands: Received" + 
                			" PACKET NUMBER ----------------> " + count);
                System.out.println("PAICommands: Received " 
                			+ new String(p.getData()) +
                        			" from " + p.getAddress().getHostAddress());
            }
        } catch (IOException ep) {
            System.out.println("PAICommands: Error opening socket");
        }
    }

    public void timerTriggered() {
        try {
                byte b[] = (new String("Hello Proteus " + 
                				String.valueOf(count)).getBytes());
                DatagramPacket p =new DatagramPacket(b, b.length, 
                					new InetAddress(sendTo), 5555);
                pai.send(s,p);
        }      catch (IOException eh) {
            System.out.println("Error Sending Data");
        }
    }


    public String command(String command, String args[]) {
        if (command.equals("init")) {
            init();
            return "OK";
        }
        else if (command.equals("setSendTo")) {
            sendTo = args[0];
            return "OK";
         }
        else if (command.equals("start")) {
            start();
            return "OK";
        }
       else if (command.equals("trigger")) {
             timerTriggered();
             return "OK";
         }
       else if (command.equals("cleanUp")) {
             pai.cleanUp();
             return "OK";
         }

        return "ERROR";
    }

    public void setPAI(PAIInterface pai) {
        this.pai=pai;
    }
}
\end{verbatim}
\normalsize

Firstly, you'll notice that \emph{PAICommands} implements three interfaces:

\begin{itemize}
\item \textbf{CommandInterface:} so that it understands how to execute
commands, as described in the previous chapter. 
\item \textbf{PAIAccessInterface:} (see pai.broker.PAIAccessInterface) 
this interface is a tagging mechanism that tells the subsystem 
that your Java object wishes to use the JNI
interface.  Without this, your object cannot use the efficient memory
allocation that the subsystem provides for managing all Java objects.
You could in principle access PAI directly but you'd have to manage 
pointers yourselves, which would be tedious.  Using this interface,
the \emph{JavaBroker} notifies you of the instance of the 
\emph{pai} interface by calling the implemented method from this
interface, called \textbf{setPAI(PAIInterface pai)}, as illustrated.  This
allows you to store the pai reference locally and use it within your Java
object.
\item \textbf{PAISocketListener:} this allows your class to be notified when
data arrives at a \emph{PAISocket}. Briefly, within Java, you attach yourself
(or attach others) as a listener on an object and this results in the 
notification of certain events when they arrive. To make the semantics
clear, you have to implement an interface which enables the source object to 
notify you when its state changes.  This is achieved generally by a listener
interface, which PAISocketListener implements. Java listeners are an implementation
of a callback mechanism.  Within C++ you have to point to actual functions,
which Java you attach listeners. The interface looks like this:
 
\footnotesize
\begin{verbatim}
package pai.event;

public interface PAISocketListener {
    public void dataReceived(PAISocketEvent event);
}
\end{verbatim}
\normalsize

\noindent which contains one method, \emph{dataReceived} that gets
invoked when data arrives at the socket. The \emph{dataReceived} method
passes a PAISocket event, which contains details about the socket that
issued the event (i.e. you may be a listener to several sockets). Once this
event is received, you can use \emph{pai} to retrieve the data, using the
\emph{receive} method - which takes the socket as a parameter and
a DatagramPacket, which is a container to hold the incoming data (this
is the standard Java mechanism for doing this).
\end{itemize}

Briefly, the object is initialized by creating a socket on port 5555. We then
add ourselves as a listener for events from this socket.   The \emph{start}
method gets invoked when a \emph{start} command is received from
the NS2 TCL script, this simply invokes the trigger function, which results
in a data packets being sent to the the \emph{sendTo} NS2 node.  The
\emph{sendTo} variable is set using the \emph{setSendTo} TCL
command as described previously.

Within the \emph{dataReceived} method, messages are printing our if
logging is enabled. There is a static class in \emph{pai.impl.Logging},
which is set globally for all classes within the JVM to turn on or off 
comments. If it is enabled then you get a verbose output - the default
is that it is set to \emph{on}.



\section{Example 2: Using the Trigger Mechansim}
\label{jnipai:trigger}

This is a Java example, which implements the \emph{ProtoApp}
scenario, the demonstration class for Protolib. Briefly, a trigger 
is set off once a second to tell the Java object to send data to
another node. When the data is received by
the receiving NS-2 node, another Java method is 
triggered allowing it to read the data using the  \emph{PAISocketListener}
interface.

The actual trigger mechanism is implemented in C++ but this then 
triggers a method in the Java object to tell it to read
the data. This example also uses the \emph{PAICommands}
class. When the C++ trigger times out, it sends a 'trigger' command
to the Java object, which results in the timerTriggered() method
being called.  This is equivalent functionality to ProtoApp, but in 
Java. However, the actual interface to the timer is set within the
NS2 TCL script and not the C++ class, enabling the programmer
to change the timer's parameters without having to recompile the
whole of NS2. 

\subsection{The TCL Side}
\label{jni:tclside}

The following is the TCL script part of the implementation, which creates two 
\emph{JavaAgent} NS2 nodes attaches the PAICommands Java object
to them, initializes them and then sets up the node that will receive the
data by invoking the \emph{setSendTo} command on the first node - node
0 sends data to node 1 in this example.  We then start a timer by using

\footnotesize
\begin{verbatim}
$ns_ at 0.0 "$p1 startTimer 1 -1"
\end{verbatim}
\normalsize

\noindent which sets off a timer that times out once per second and runs
forever (i.e. -1 flag).  The timer is stopped at the end of the
simultation. Here is the TCL script:

\footnotesize
\begin{verbatim}
# Create simulator instance
set ns_ [new Simulator]

# Create two nodes
set n1 [$ns_ node]
set n2 [$ns_ node]

# Put a link between them
$ns_ duplex-link $n1 $n2 64kb 100ms DropTail
$ns_ queue-limit $n1 $n2 100
$ns_ duplex-link-op $n1 $n2 queuePos 0.5
$ns_ duplex-link-op $n1 $n2 orient right

puts "Creating PAI Broker Agents ..."   
# Create two Protean example agents and attach to nodes
set p1 [new Agent/JavaAgent]
$ns_ attach-agent $n1 $p1

set p2 [new Agent/JavaAgent]
$ns_ attach-agent $n2 $p2

puts "CREATED OK          ....... ..." 
    
# Initialize each broker telling it what its NS2 address is

puts "In script: Initializing  ..." 
	
$ns_ at 0.0 "$p1 initAgent"
$ns_ at 0.0 "$p2 initAgent"

$ns_ at 0.0 "$p1 setClass 
/Users/scmijt/Apps/nrl/p2ps-ns2/classes pai.examples.ns2.PAICommands"
$ns_ at 0.0 "$p2 setClass 
/Users/scmijt/Apps/nrl/p2ps-ns2/classes pai.examples.ns2.PAICommands"

puts "Starting simulation ..." 

$ns_ at 0.0 "$p1 javaCommand init"
$ns_ at 0.0 "$p2 javaCommand init"

$ns_ at 0.0 "$p1 javaCommand setSendTo [$n2 node-addr]"

$ns_ at 0.0 "$p1 javaCommand start"

# The timer is started within C++ code NOT Java but the
# parameters are specified here

$ns_ at 0.0 "$p1 startTimer 1 -1"

# Stop
$ns_ at 9.0 "$p1 stopTimer"
$ns_ at 9.0 "$p2 stopTimer"

#Clean up objects

$ns_ at 10.0 "$p1 cleanUp"
$ns_ at 10.0 "$p2 cleanUp"

$ns_ at 10.0 "finish $ns_"

proc finish {ns_} {
$ns_ halt
delete $ns_
}

$ns_ run
\end{verbatim}
\normalsize

This example, will run the timer once per second (well, NS2 second anyway - 
which is non real-time so in effect one second will be microseconds ....) and
iterate for 10 iterations as specified by the NS2 time-stepping as shown.


\section{Example 3: Sending Data Using Multicast}
\label{jni:multicast}

A Java example, which also implements the ProtoApp
scenario but this timer uses a multicast address
to send the data between the nodes.  The first node
sends the data to the multicast address and the second
node listens to this address and gets notified when 
something happens.  This example uses the
\emph{pai.examples.ns2.MulticastTimerDemo} Java class
to implement the Java functionality.


\subsection{The TCL Side}
\label{jni:tclside}

The following is the TCL script part of the implementation, which creates
a muticast enabled NS2 and creates a multicast address for 
communication.  The multicast address to be used must be specified in 
NS2 and then passed to the Java objects so they know which address 
to use i.e. by using the \emph{setGroupAddress} java TCL script 
command as illustrated below. Two \emph{JavaAgent} NS2 nodes are
created and attach a \emph{MulticastTimerDemo} object: 

\footnotesize
\begin{verbatim}
# Create multicast enabled simulator instance
set ns_ [new Simulator -multicast on]
$ns_ multicast

# Create two nodes
set n1 [$ns_ node]
set n2 [$ns_ node]

# Put a link between them
$ns_ duplex-link $n1 $n2 64kb 100ms DropTail
$ns_ queue-limit $n1 $n2 100
$ns_ duplex-link-op $n1 $n2 queuePos 0.5
$ns_ duplex-link-op $n1 $n2 orient right

# Configure multicast routing for topology
set mproto DM
set mrthandle [$ns_ mrtproto $mproto  {}]
 if {$mrthandle != ""} {
     $mrthandle set_c_rp [list $n1]
}

# 5) Allocate a multicast address to use
set group [Node allocaddr]
   
puts "Creating Java Broker Agents ..."   
# Create two Protean example agents and attach to nodes
set p1 [new Agent/JavaAgent]
$ns_ attach-agent $n1 $p1

set p2 [new Agent/JavaAgent]
$ns_ attach-agent $n2 $p2

puts "CREATED OK          ....... ..." 
    
# Initialize C++ agents

puts "In script: Initializing  ..." 
	
$ns_ at 0.0 "$p1 initAgent"
$ns_ at 0.0 "$p2 initAgent"

#set up the class

$ns_ at 0.0 "$p1 setClass 
/Users/scmijt/Apps/nrl/p2ps-ns2/classes pai.examples.ns2.MulticastTimerDemo"
$ns_ at 0.0 "$p2 setClass 
/Users/scmijt/Apps/nrl/p2ps-ns2/classes pai.examples.ns2.MulticastTimerDemo"

puts "Starting simulation ..." 

$ns_ at 0.0 "$p1 javaCommand setGroupAddress $group"

$ns_ at 0.0 "$p1 javaCommand init"
$ns_ at 0.0 "$p2 javaCommand init"

$ns_ at 0.0 "$p1 javaCommand start"

# The timer is started within C++ code NOT Java but the
# parameters are specified here

$ns_ at 0.0 "$p1 startTimer 1 -1"

# Stop
$ns_ at 9.0 "$p1 stopTimer"
$ns_ at 9.0 "$p2 stopTimer"

#Clean up objects 

$ns_ at 10.0 "$p1 cleanUp"
$ns_ at 10.0 "$p2 cleanUp"

$ns_ at 10.0 "finish $ns_"

proc finish {ns_} {
$ns_ halt
delete $ns_
}

$ns_ run
\end{verbatim}
\normalsize

We then initialize the two \emph{JavaAgent} NS2 nodes
start the first node. This results in the first node sending a data
packet to the chosen Multicast address, which results in the second node
receiving notification of this transfer. The timer is then kicked off,
which repeats this process 10 times

\subsection{The Java Side}
\label{jni:javaside}

On the java side \emph{MulticastTimerDemo.java} implements various
commands, rather similar to the PAICommands class, except that it
replaces the \emph{setSender} function with the Multicast address,
enabling all nodes to talk to a central address. This enables nodes to
automatically send data to collections of nodes and it is this process
that will enable P2PS to discover the address of other nodes using
its discovery mechanisms. The code looks like this:

\footnotesize
\begin{verbatim}
package pai.examples.ns2;

import pai.broker.CommandInterface;
import pai.broker.PAIAccessInterface;
import pai.broker.JavaBroker;
import pai.api.PAIInterface;
import pai.net.DatagramSocket;
import pai.net.DatagramPacket;
import pai.net.InetAddress;
import pai.net.MulticastSocket;
import pai.impl.PAITimer;
import pai.impl.Logging;
import pai.event.PAISocketEvent;
import pai.event.PAISocketListener;

import java.net.SocketException;
import java.io.IOException;

/**
 * @author Ian Taylor.
 * A demo of a NS2 Java Object that
 */
public class MulticastTimerDemo implements CommandInterface, PAIAccessInterface,
                                        PAISocketListener {
    PAIInterface pai;
    MulticastSocket s;
    PAITimer t;
    int count=0;

    public void init() {

        try {
            s = new MulticastSocket(5555);
            pai.addPAISocketListener(s,this);
            pai.joinGroup(s, 
            		new InetAddress(JavaBroker.getMulticastAddress()));
        } catch (SocketException e) {
            System.out.println("Error opening socket");
        }
        catch (IOException ep) {
            System.out.println("Error opening socket");
        }
    }

    void start() {
   	    timerTriggered();  // transmit first packet right away
        }

    public void dataReceived(PAISocketEvent sv) {
        try {
            System.out.println("Receiving ----------------------------");
            ++count;
            byte b[] = new byte[15];
            DatagramPacket p = new DatagramPacket(b,b.length);
            pai.receive(s, p);
            if (Logging.isEnabled()) {
                System.out.println("PAICommands: Received " + 
                			"PACKET NUMBER ----------------> " + count);

                System.out.println("PAICommands: Received " 
                			+ new String(p.getData()) +
                        			" from " + p.getAddress().getHostAddress());
            }
        } catch (IOException ep) {
            System.out.println("PAICommands: Error opening socket");
        }
    }

    public void timerTriggered() {
        try {
            byte b[] = (new String("Hello Proteus " + 
                					String.valueOf(count)).getBytes());
            System.out.println("Address is " +  
                			JavaBroker.getMulticastAddress());
            DatagramPacket p =new DatagramPacket(b, b.length, 
                 new InetAddress(JavaBroker.getMulticastAddress()), 5555);
            pai.send(s,p);
        }      catch (IOException eh) {
            System.out.println("Error Sending Data");
        }
    }
    
    public String command(String command, String args[]) {
        if (command.equals("init")) {
            init();
            return "OK";
        }
        else if (command.equals("start")) {
            start();
            return "OK";
        }
       else if (command.equals("trigger")) {
             timerTriggered();
             return "OK";
         }
       else if (command.equals("cleanUp")) {
             pai.cleanUp();
             return "OK";
         }
        return "ERROR";
    }

    public void setPAI(PAIInterface pai) {
        this.pai=pai;
    }
}
\end{verbatim}
\normalsize


The first thing to notice here is that we are not using the \emph{PAI}
Java interface to create our Multicast socket, but we are using the
\emph{MulticastSocket} class.  The \emph{MulticastSocket} class
we are using here is the PAI re-implementation of the java.io.MulticastSocket 
class for use with our Java PAI interface.  The actual implementation
of \emph{MulticastSocket} simply calls the PAI interface in order to 
create the appropriate socket, that is in this case, it creates a
normal DatagramSocket by using the default constructor and sets
Multicast to \emph{true} on this socket so that it can join the 
multicast group address.

The actual multicast group address being used is set from the
TCL script, as described.  Java NS2 object gain access to this
address by using the:

\footnotesize
\begin{verbatim}
           JavaBroker.getMulticastAddress();
\end{verbatim}
\normalsize

\noindent static method call.  This enables any Java object 
within this JVM to gain access to the default Multicast address 
that it should use.  P2PS uses this same address also when 
communicating with other P2PS nodes, as we will see in Chapt.
\ref{paip2ps}.   Here therefore, we join the Multicast group
by issuing the following PAI command:

\footnotesize
\begin{verbatim}
            pai.joinGroup(s, new InetAddress(JavaBroker.getMulticastAddress()));
\end{verbatim}
\normalsize

The rest of the class simply implements the same functionality
as the PAICommands class discuss earlier in this chapter.

\section{Conclusion}

In this chapter, the Java PAI interface was discussed. An overview
of the architecture was given and a brief description of how the
classes implement this functionality. We then outlined three
examples, which show how one would send data between NS2 
nodes, how one would use the timing interface to send repeated
calls and how one would use a Multicast address to send data to 
any nodes that are listening to this address.
